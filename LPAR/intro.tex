Mathematical results that require extensive computational proofs have often raised suspicion amongst mathematicians. 
A landmark example is the \emph{four-color theorem}, which states that any planar graph can be colored with at most four colors. 
In 1879, Alfred Kempe published a proof of the four-color theorem, which was discovered to be incorrect 11 years later by Headwood. 
The main idea of Kempe's proof was salvaged by Headwood to prove the weaker \emph{five-color theorem}. 
A full proof of the four-color theorem was not found until 1976, when Kenneth Appel and Wolfgang Haken used a computer to verify the 1834 cases in which they proved that a minimal counterexample must exist.
The proof of Appel and Haken was controversial, as some small errors had been found in their initial calculations of 1976. Finally, in 2005, Georges Gonthier formalized a full proof of the four-color theorem in the \textsf{Coq} proof assistant, thus culminating with any lingering doubts.


