Mathematicians are often skeptical of proofs backed by extensive computation.
A landmark example is the \emph{four-color theorem}, which states that any planar graph can be colored with at most four colors. 
In 1879, Alfred Kempe published a proof of the four-color theorem, which was discovered to be incorrect 11 years later by Headwood. 
The main idea of Kempe's proof was salvaged by Headwood to prove the weaker \emph{five-color theorem}. 
A full proof of the four-color theorem was not found until 1976, when Kenneth Appel and Wolfgang Haken used a computer to verify the 1,834 cases that provably could contain a counterexample.
Their proof was controversial, and indeed, small errors have been found in their initial calculations.
Finally, in 2005, Georges Gonthier formalized a full proof of the four-color theorem in the \textsf{Coq} proof assistant, thus laying to rest any lingering doubts about the correctness of a computational proof.