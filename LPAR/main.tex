\PassOptionsToPackage{dvipsnames}{xcolor}

\documentclass{easychair}
\usepackage[T1]{fontenc}
\usepackage[utf8]{inputenc}
\usepackage{listings}

\usepackage{doc}
\usepackage{mathtools}
\usepackage{pgfplots}
\usepackage{pgfmath}
\usepackage{float}
\usepackage[misc]{ifsym}
\usepackage[ruled]{algorithm2e}
\usepackage{subcaption}
\captionsetup{compatibility=false}
\usepackage{graphicx}
\usepackage{mathtools}
\usepackage{amsthm}
\usepackage{amsfonts}
\usepackage{amsmath}
\usepackage{amssymb}
\usepackage{todonotes}
\usepackage{nicefrac}
\usepackage{cleveref}
\usepackage{booktabs}
\usepackage{multirow}
\usepackage{xpatch}
\usepackage{appendix}
\usepackage{adjustbox}
\usepackage{interval}

\usepackage{tikz}
\usetikzlibrary{shapes.geometric}
\usetikzlibrary{fit}
\usetikzlibrary{arrows}
\usetikzlibrary{patterns}
\usetikzlibrary{matrix}
\usetikzlibrary{positioning}
\usetikzlibrary{backgrounds}

\usepackage{color}
\definecolor{keywordcolor}{rgb}{0.7, 0.1, 0.1}   % red
\definecolor{tacticcolor}{rgb}{0.0, 0.1, 0.6}    % blue
\definecolor{commentcolor}{rgb}{0.4, 0.4, 0.4}   % grey
\definecolor{symbolcolor}{rgb}{0.0, 0.1, 0.6}    % blue
\definecolor{sortcolor}{rgb}{0.1, 0.5, 0.1}      % green
\definecolor{attributecolor}{rgb}{0.7, 0.1, 0.1} % red

\def\lstlanguagefiles{lstlean.tex}
% set default language
\lstset{language=lean}

\makeatletter
\def\orcidID#1{\href{http://orcid.org/#1}{\protect\raisebox{-1.25pt}{\protect\includegraphics{orcid_color.eps}}}}
\makeatother

\newtheorem{theorem}{Theorem}
\newtheorem{lemma}{Lemma}

\title{A Lean Verification of the Empty Hexagon Theorem}%
% \thanks{Other people who contributed to this document include Maria Voronkov
%   (Imperial College and EasyChair) and Graham Gough (The University of
%   Manchester).}}

% Authors are joined by \and. Their affiliations are given by \inst, which indexes
% into the list defined using \institute
%
\author{ Cayden Codel \orcidID{0000-0003-2295-1299}
   \and  James Gallicchio \orcidID{0000-0003-2295-1299}
   \and  Wojciech Nawrocki \orcidID{0000-0003-2295-1299}
  \and  \\ Bernardo Subercaseaux  \orcidID{0000-0003-2295-1299} }
%
\institute{
  Carnegie Mellon University, Pittsburgh, PA 15213, USA\\
  \email{\{ccodel, jcallicc, wnawrock, bsuberca\}@andrew.cmu.edu}
 }


%  \authorrunning{} has to be set for the shorter version of the authors' names;
% otherwise a warning will be rendered in the running heads. When processed by
% EasyChair, this command is mandatory: a document without \authorrunning
% will be rejected by EasyChair

\authorrunning{Codel, Gallicchio, Nawrocki, and Subercaseaux}

% \titlerunning{} has to be set to either the main title or its shorter
% version for the running heads. When processed by
% EasyChair, this command is mandatory: a document without \titlerunning
% will be rejected by EasyChair
\titlerunning{A Lean Verification of the Empty Hexagon Theorem}

\begin{document}

\maketitle

\begin{abstract}
  A recent breakthrough in computer-assisted mathematics proved that every set of $30$ points contains an empty hexagon, 
  thus closing a line of research opened by Erd\H{o}s and Szekeres in 1935. The proof combines geometric insights with automated reasoning techniques, 
 resulting in a sophisticated propositional encoding of the problem with $O(n^4)$ many clauses (where $n$ ends up being $30$) that is then solved using a SAT solver and parallel computation.
 This paper presents a formalization of the proof in the Lean theorem prover, thus certifying its correctness. To achieve this we have formalized several ideas of both discrete computational geometry and SAT-encodings that have been successfully applied in different Erd\H{o}s-Szekeres type problems.
We hope this work sets a new standard for verification when extensive computation is used for discrete geometry problems, and increases the trust of the community in the results obtained by these methods.

\end{abstract}

\section{Introduction}
\label{sec:intro}
Mathematicians are often skeptical of proofs relying on extensive computation.
A landmark example is the \emph{four-color theorem}, which states that any planar graph can be colored with at most four colors. 
In 1879, Alfred Kempe published a proof of the four-color theorem, which was discovered to be incorrect 11 years later by Headwood~\cite{Walters2004ItAT,Wilson2002GraphsCA}.\footnote{The main idea of Kempe's proof was salvaged by Headwood to prove the weaker \emph{five-color theorem}, and it is still a building block in the theory of planar colorings~\cite{Walters2004ItAT}.} 
A full proof of the four-color theorem was not found until 1976, when Kenneth Appel and Wolfgang Haken used a computer to check the 1\,834 cases that might contain a counterexample~\cite{appelFourColorProblem1978}.
Their proof was controversial, and indeed, small errors were found in their initial calculations~\cite{Walters2004ItAT,Wilson2002GraphsCA}.
In 2005, Georges Gonthier formalized a full proof of the four-color theorem in the \textsf{Coq} proof assistant~\cite{gonthierFourColourTheorem2008a}, thus laying to rest any lingering doubts about the correctness of the result.
To this day, all proofs known for this theorem require the assistance of computers.

The four-color theorem is far from the end of the story when it comes to computer-assisted mathematics.
An emerging trend is to use SAT solvers (or other automated reasoning tools) to prove mathematical theorems by analyzing finite objects~\cite{avigad2023mathematics}. 
To name a few examples, the Erd\H{o}s Discrepancy Conjecture~\cite{konev2014sat}, Keller's conjecture~\cite{brakensiek2023resolution}, the Packing Chromatic number of the infinite grid~\cite{Subercaseaux_Heule_2023}, and the Pythagorean Triples Problem~\cite{Heule_2016} were all resolved using SAT solvers.
All such SAT-based results follow a common structure: in order to show that a mathematical theorem $\mathcal{T}$ holds, one proves the following:
\begin{enumerate}
  \item (\textbf{Reduction Theorem}) There is a finite object $\mathcal{O}$ such that either (\emph{positive case}) if $\mathcal{O}$ exists then $\mathcal{T}$ holds, or (\emph{negative case}) if $\mathcal{O}$ does not exist then $\mathcal{T}$ holds.\footnote{Note that $\mathcal{T}$ can be a statement concerning infinite objects, as for example in the case of Keller's conjecture~\cite{brakensiek2023resolution}.}
  \item (\textbf{Encoding Theorem}) There is a propositional formula $\varphi_{\mathcal{O}}$ such that either (\emph{positive case}) if $\varphi_{\mathcal{O}}$ is satisfiable then $\mathcal{O}$ exists, or (\emph{negative case}) if $\varphi_{\mathcal{O}}$ is unsatisfiable then $\mathcal{O}$ does not exist.
  \item (\textbf{SAT Result}) In the \emph{positive case}, a SAT solver finds a satisfying assignment for $\varphi_{\mathcal{O}}$, and in the \emph{negative case}, a SAT solver finds that $\varphi_{\mathcal{O}}$ is unsatisfiable.
\end{enumerate}

Unfortunately, the formula $\varphi_{\mathcal{O}}$ concerning the existence of $\mathcal{O}$ is often too large or computationally hard to be solved directly, so a fourth step is added to the above method: a \textbf{Re-encoding Theorem} showing that $\varphi_{\mathcal{O}}$ is equisatisfiable with a simpler formula $\phi_{\mathcal{O}}$.
All the examples we mentioned above add this step, and so will we.

One of the benefits of using a SAT solver for mathematical proofs is that in the \emph{positive case}, it is usually possible to construct the desired object $\mathcal{O}$ from a satisfying assignment to $\phi_{\mathcal{O}}$, and then by exhibiting $\mathcal{O}$, the correctness of theorem $\mathcal{T}$ can be easily verified by humans.
The \emph{negative case}, however, is much harder to trust, as the unsatisfiability proofs emitted by SAT solvers are usually too large for human verification, and depend crucially on the correctness of the encoding.\footnote{In the positive case, the correctness of the encoding is not even required, as an incorrect encoding might still lead to obtaining a concrete object $\mathcal{O}$ that satisfies the desired properties, proving the intended theorem regardless.}
% \todo[inline]{CC: I don't like how this paragraph is structured. The negative case isn't inherently \emph{harder}. Instead, the method of verifying the overall result is different: rather than constructing an actual object we can see or compute on, we have to rely on a non-existence result. Thus, the reduction must be correct, as errors can't be caught the same way they can be in a physical construction.\\ BS: I disagree; the NP vs coNP asymmetry is exactly about how easy it is to convince a verifier of existence vs. non-existence results.}
For example, the resolution of the Boolean Pythagorean triples problem required an unsatisfiability proof of 200 terabytes, the largest known to date~\cite{Heule_2016,lambTwohundredterabyteMathsProof2016}.
This raises a fundamental question for computational mathematics:
\begin{center}
  \emph{How can we trust the correctness of a theorem $\mathcal{T}$ that relies on a very long unsatisfiability proof?}
\end{center}
\begin{figure}[ht]
    \centering
    \begin{tikzpicture}
      \node[draw, rounded corners] (theorem) at (0,0) {Theorem $\mathcal{T}$};
      \node[draw, rounded corners] (object) at (9,0) { Finite Object $\mathcal{O}$};
      \node[draw, rounded corners, align=center, fill=green!40!white] (tiffo) at (4.5,0) { \textbf{Reduction theorem}\\$\mathcal{T}$ holds if $\mathcal{O}$ does not exist};
      \draw[->, thick] (tiffo) -- (theorem);
      \draw[->, thick] (tiffo) -- (object);
      \node[draw, rounded corners, align=center] (varphi) at (9, -3) {CNF formula $\varphi_{\mathcal{O}}$};
      \node[draw, rounded corners, align=center, fill=yellow!40!white] (encoding) at (9, -1.5) { \textbf{Encoding theorem}\\If $\varphi_{\mathcal{O}}$ is unsatisfiable then $\mathcal{O}$ does not exist};
      \draw[->, thick] (encoding) -- (object);
      \draw[->, thick] (encoding) -- (varphi);
      \node[draw, rounded corners] (phi) at (0, -3) {CNF formula $\phi_{\mathcal{O}}$};
      \node[draw, rounded corners, align=center, fill=blue!20!white] (equisat) at (4.5, -3) { \textbf{Reencoding theorem}\\$\phi_{\mathcal{O}}$ is equisatisfiable to $\varphi_{\mathcal{O}}$};
      \draw[->, thick] (equisat) -- (phi);
      \draw[->, thick] (equisat) -- (varphi);
      \node[draw, rounded corners] (solver) at (0, -6) {SAT Solver $\mathcal{S}$};
      \node[draw, rounded corners] (unsat-proof) at (4.5, -4.75) {Unsatisfiability proof $\Gamma$};
      \node[circle] (solverphi) at (1.5, -4.75) {$\mathcal{S}(\phi_{\mathcal{O}})$};
      \draw[-, thick] (solver) -- (solverphi);
      \draw[-, thick] (phi) -- (solverphi);
      \draw[->, thick] (solverphi) -- (unsat-proof);
      \node[draw, rounded corners] (checker) at (9, -4.75) {Proof checker $\mathcal{C}$};
      \node[circle, draw, fill=purple!30!white] (checkerphi) at (7.125, -6) {$\mathcal{C}(\Gamma)$};
      \draw[->, thick] (unsat-proof) -- (checkerphi);
      \draw[->, thick] (checker) -- (checkerphi);
      \node[draw, rounded corners, align=center, fill=red!20!white] (checkerCorrectness) at (9, -7.5) { \textbf{Checker verification theorem}\\$\mathcal{C}$ is correct};

      \draw[->, thick] (checkerCorrectness) -- (checker);
      \draw[red, dashed, thick, rounded corners] (-1.75,1) -- (-1.75, -4) -- (12.75, -4) -- (12.75, 1) -- cycle;
    \end{tikzpicture}
    \caption{General structure of the verification pipeline for a SAT-based theorem in the \emph{negative case}. The dashed rectangle encloses the main focus of this paper, whereas for the rest of the proof we leverage already existing tools.}\label{fig:proof-structure}
  \end{figure}

The main contribution of this article is to provide the first example of a formally-verified proof of such a theorem in the context of discrete geometry, and the first one overall in Lean, thus addressing different aspects of the question above.
Our proof pipeline involves several components, as illustrated in~\Cref{fig:proof-structure} and described in the rest of the paper. All our code is publicly available at \url{https://github.com/bsubercaseaux/EmptyHexagonLean}.
% \footnote{CC: I'm not sure this characterization is correct. For example, another group of researchers verified in Coq the encoding used in the Pythagorean Triples problem, thus verifying that result. One could say that our contribution is the first \emph{end-to-end} verification, but I don't see how it's inherently distinct from what this other group did with the Pythagorean Triples problem.}
% BS: Thanks Cayden, you were right.
\paragraph{The Empty Hexagon Number.}
In the 1930s, Stein, Erd\H{o}s, and Szekeres mixed geometry and Ramsey theory by studying how many points in the plane in \emph{general position} (i.e., no three points on a common line) are required for a convex $k$-gon to always appear. 
%Let $g(k)$ be the minimum number of points in general position to force a convex $k$-gon.
%The celebrated Erd\H{o}s-Szekeres theorem states that $g(k) \leq \binom{2k-4}{k-2} + 1$~\cite{erdosCombinatorialProblemGeometry2009}. 
%Moreover, Erd\H{o}s and Szekeres conjectured that $g(k) = 2^{k-2} + 1$, which has only been proven for $k \leq 6$.
In a similar spirit, Erd\H{o}s defined $h(k)$ as the minimum number of points in general position that is guaranteed to contain a $k$-hole (i.e., a convex $k$-gon without any other point inside).
It is easy to check that $h(3) = 3$ and $h(4) = 5$. In 1978, Harborth established that $h(5) = 10$~\cite{Harborth1978}, and in a surprising turn of events, Horton proved in 1983 that $h(7) = \infty$, meaning one could always avoid $7$-holes~\cite{hortonSetsNoEmpty1983}. 
The only case left open was thus $h(6)$.

The \emph{Empty Hexagon Theorem}, establishing $h(6)$ to be finite, was proven independently by Gerken and Nicolás in 2006~\cite{gerkenEmptyConvexHexagons2008,nicolasEmptyHexagonTheorem2007}, and then refined by Valtr in 2008~\cite{valtr}.
Yet the known range of values for $h(6)$ was quite large, with $29 \leq h(6) \leq 1717$, until Heule and Scheucher used a SAT solver in a recent breakthrough to prove that $h(6) = 30$~\cite{emptyHexagonNumber}, a result we refer to as ``\emph{The Empty Hexagon Number}.''
Now that all the values of $h$ are known, and especially given the extensive computation involved in its proofs, we believe that the final chapter in the story begun by Stein, Erd\H{o}s, and Szekeres is a formal verification of the Empty Hexagon Number.

\paragraph{Related Work.} Despite the crucial role of formal verification in the SAT community (e.g., verified solvers~\cite{oeVersatVerifiedModern2012,skotam_creusat_2022}, proof checking~\cite{lammichEfficientVerifiedSAT2020,tanVerifiedPropagationRedundancy2023}, etc.), the end-to-end formal verification of mathematical proofs obtained through SAT-solving is very recent. To the best of our knowledge, the first and only instance of a formally verified SAT-based mathematical proof before our work corresponds to the Pythagorean Triples Problem, verified in the \textsf{Coq} proof assistant by Cruz-Filipe and Schneider-Kamp~\cite{formalPythagoreanTriples,LPAR-21:Formally_Proving_Boolean_Pythagorean}. In terms of verification of SAT encodings in Lean, the work of Codel, Avigad and Heule pioneered~\cite{Cayden}, while other encoding libraries existed in different frameworks, such as the work of Giljeg\r{a}rd and Wennerbreck in \textsf{CakeML}~\cite{GilAndWennerbeck}, which allowed for verifying SAT-based solutions to different math puzzles (e.g., Sudoku, Kakuro, the \emph{N-queens} problem, etc.).


\paragraph{Lean.} Kickstarted by Leonardo de Moura in 2013~\cite{demouraLeanTheoremProver2015}, the Lean theorem prover has arguably become the most popular for formalizing modern breakthroughs in mathematical research. 
On the one hand, the recent success of projects such as the~\emph{Liquid Tensor Experiment}~\cite{Castelvecchi2021}, or the proof of~the polynomial Freiman–Ruzsa conjecture~\cite{gowers2023conjecture, slomanATeamMathProves2023} has brought significant attention to~Lean as an interactive theorem prover.
On the the other hand, the \textsf{mathlib} project~\cite{The_mathlib_Community_2020} has already cemented the basics of many areas of mathematics, thus allowing modern results to be formalized much more easily by relying on hundreds of thousands of existing lines of code. In this spirit, we connect our formalization to~\textsf{mathlib} as much as possible.

\paragraph{Outline of the paper.} \Cref{sec:triple-orientations} discusses the basic geometric aspects of the problem, and in particular, \emph{triple orientations} (also known as \emph{signotopes}), a fundamental tool in computational geometry to represent discretely problems involving an a priori continuous space (i.e., $\mathbb{R}^2$), which is the basis of Heule and Scheucher's SAT encoding. Then,~\Cref{sec:symmetry-breaking} deals with two assumptions that are key to break symmetries in the problem and thus make the SAT encoding practically feasible. Namely, that one can assume without loss of generality the following two properties at the same time: (i) points are labeled from left to right without two of them having the same $x$-coordinate, and (ii) the triples $(p_1, p_i, p_j)$ are always oriented counterclockwise for $i < j$. 
\Cref{sec:leansat} presents the \textsf{LeanSAT} library, which plays a key role in proving the correctness of encodings. Next,~\Cref{sec:empty-triangle} presents how the previous elements are already enough to formalize a SAT-based proof for the \emph{Empty Triangle Theorem}, a much simpler variant involving only triangles that does not require reencodings. \Cref{sec:empty-hexagon-number} presents the details of the formalization of $h(6) = 30$, the Empty Hexagon Number. We conclude by discussing the next steps towards the formal verification of other Erd\H{o}s-Szekeres-type problems in~\Cref{sec:conclusion}.



\section{The Sortedness Assumption and Symmetry Breaking}
\label{sec:symmetry-breaking}
\emph{Symmetry breaking} plays a key role in SAT solving by reducing the search space of satisfying assignments for a formula~\cite{biereHandbookSatisfiabilityVolume2009,Crawford},
thus making a wider range of formulas practical to solve.
For example, if one proves that all satisfying assignments to a formula $\phi$ have either (i) $x_1 = 0, x_2 = 1$, or  (ii) $x_1 = 1, x_2 = 0$, and that there is a bijection between satisfying assignments of forms (i) and (ii),
then one can assume, \emph{without loss of generality}, that $x_1 = 0, x_2 = 1$, and thus add unit clauses $\ov{x_1}$ and $x_2$ to the formula $\phi$ while preserving its satisfiability.
There are several techniques that can automatically find symmetry-breaking clauses,
such as structured bounded variable addition~\cite{sbva},
but it is accepted wisdom in the SAT-solving community that problem-specific symmetry breaking is more effective.

In their proof of the Empty Hexagon Number,
Heule and Scheucher showed that for any list of points in general position,
there exists a list of points in \emph{canonical position} with the same triple-orientations.
Canonical position is defined as follows.
\begin{definition}[Canonical Position]
A list of points~$L = (p_1,\ldots, p_{n})$ is said to be in \emph{canonical position} if it satisfies all the following properties:
\begin{itemize}
    \item \textbf{(General Position)} No three points are collinear, i.e., for all $1 \leq i < j < k \leq n$, we have $\sigma(p_i, p_j, p_k) \neq 0$.
    \item \textbf{($x$-order)} The points are sorted with respect to their $x$-coordinates, i.e., $x(p_i) < x(p_j)$ for all $1 \leq i < j \leq n$.
    \item \textbf{(CCW-order)} All orientations $\sigma(p_1, p_i, p_j)$, with $1 < i < j \leq n$, are counterclockwise.
    \item \textbf{(Lex-order)} The list of orientations \( \left(\sigma\left(p_{\lceil \frac{n}{2} \rceil -1}, p_{\lceil \frac{n}{2} \rceil},p_{\lceil \frac{n}{2} \rceil+1}\right), \ldots, \sigma\left(p_2, p_3, p_4\right) \right)\) is not lexicographically smaller than the list \(\left(\sigma\left(p_{\lfloor \frac{n}{2} \rfloor  + 1}, p_{\lfloor \frac{n}{2} \rfloor+2},p_{\lfloor \frac{n}{2} \rfloor+3}\right), \ldots, \sigma\left(p_{n-2}, p_{n-1}, p_{n}\right) \right).\)
    % Given the general position condition, all orientations are in $\{-1, 1\}$, and thus the lexicographic condition is equivalent to stating that there is an index $i$ such that $\forall j < i$, $\textsf{Left}[j] = \textsf{Right}[j]$, and $\textsf{Left}[i] = -1$ but $\textsf{Right}[i] = 1$.
\end{itemize}
\end{definition}

The three ordering properties each break a different symmetry.
First, the $x$-order property breaks symmetry due to how we label the points by ensuring that the points are labeled from left to right.
The $x$-order property also simplifies the encoding of clauses~\labelcref{eq:insideClauses1,eq:insideClauses2,eq:holeDefClauses1,eq:signotopeClauses11,eq:signotopeClauses12},
as they rely on the points being sorted.
Second, the CCW-order property breaks symmetry due to \emph{rotation} by fixing the orientations involving the leftmost point~$p_1$.

\begin{figure}
    \centering
    \begin{subfigure}{0.24\linewidth}
        \centering
        \begin{tikzpicture}
            \node[draw, circle, black, fill=blue, inner sep=0pt, minimum size=5pt, label={[xshift=0.1cm, yshift=0.1cm]$a$}] (a) at (0*0.75,0*0.75) {};
            \node[draw, circle, black, fill=blue, inner sep=0pt, minimum size=5pt, label={[xshift=0.1cm, yshift=0.1cm]$b$}] (b) at (1.75*0.75,1*0.75) {};
            \node[draw, circle, black, fill=blue, inner sep=0pt, minimum size=5pt, label={[xshift=0.1cm, yshift=0.1cm]$c$}] (c) at (3*0.75,0.4*0.75) {};
            \node[draw, circle, black, fill=blue, inner sep=0pt, minimum size=5pt, label={[xshift=-0.3cm, yshift=-0.6cm]$d$}] (d) at (2.6*0.75,-0.4*0.75) {};
            \coordinate (a) at (0*0.75,0*0.75);
            \coordinate (b) at (1.75*0.75,1*0.75);
            \coordinate (c) at (3*0.75,0.4*0.75);
            \coordinate (d) at (2.6*0.75,-0.4*0.75);
            \fill[blue, opacity=0.3] (a) -- (b) -- (c) -- (d) -- (a) -- cycle;
        \end{tikzpicture}
        \caption{}\label{fig:equiv-a}
    \end{subfigure}
    \begin{subfigure}{0.24\linewidth}
        \centering
        \begin{tikzpicture}[scale=0.75]
            \node[draw, circle, black, fill=blue, inner sep=0pt, minimum size=5pt, label={[xshift=0.1cm, yshift=-0.6cm]$a'$}] (a) at (0*0.75,0*0.75) {};
            \node[draw, circle, black, fill=blue, inner sep=0pt, minimum size=5pt, label={[xshift=0.1cm, yshift=0.1cm]$b'$}] (b) at (-1.75*0.75,1*0.75) {};
            \node[draw, circle, black, fill=blue, inner sep=0pt, minimum size=5pt, label={[xshift=0.1cm, yshift=0.1cm]$c'$}] (c) at (3*0.75,1.4*0.75) {};
            \node[draw, circle, black, fill=blue, inner sep=0pt, minimum size=5pt, label={[xshift=-0.3cm, yshift=0.1cm]$d'$}] (d) at (0.2*0.75,2*0.75) {};
            \coordinate (a) at (0*0.75,0*0.75);
            \coordinate (b) at (-1.75*0.75,1*0.75);
            \coordinate (c) at (3*0.75,1.4*0.75);
            \coordinate (d) at (0.2*0.75,2*0.75);*0.75
            \fill[blue, opacity=0.3] (b) -- (d) -- (c) -- (a) -- (b) -- cycle;
        \end{tikzpicture}
        \caption{}\label{fig:equiv-b}
    \end{subfigure}
%    \vspace{0.5cm}
    \begin{subfigure}{0.24\linewidth}
        \centering
        \begin{tikzpicture}[scale=.75]
            \node[draw, circle, black, fill=blue, inner sep=0pt, minimum size=5pt, label={[xshift=0.1cm, yshift=-0.6cm]$a$}] (a) at (1*2.5*0.75,6*0.4*0.75) {};
            \node[draw, circle, black, fill=blue, inner sep=0pt, minimum size=5pt, label={[xshift=0.1cm, yshift=0.1cm]$b$}] (b) at (2*2.5*0.75,6.5*0.4*0.75) {};
            \node[draw, circle, black, fill=blue, inner sep=0pt, minimum size=5pt, label={[xshift=0.1cm, yshift=-0.6cm]$c$}] (c) at (2.1*2.5*0.75,1.5*0.4*0.75) {};
            \node[draw, circle, black, fill=blue, inner sep=0pt, minimum size=5pt, label={[xshift=0.1cm, yshift=-0.6cm]$d$}] (d) at (2.25*2.5*0.75,5*0.4*0.75) {};
            \node[draw, circle, black, fill=blue, inner sep=0pt, minimum size=5pt, label={[xshift=0.3cm, yshift=-0.4cm]$e$}] (e) at (2.8*2.5*0.75,10*0.4*0.75) {};
            \node[draw, circle, black, fill=blue, inner sep=0pt, minimum size=5pt, label={[xshift=0.2cm, yshift=-0.6cm]$f$}] (f) at (3.1*2.5*0.75,4.8*0.4*0.75) {};
            \coordinate (a) at (1*2.5*0.75,6*0.4*0.75);
            \coordinate (b) at (2*2.5*0.75,6.5*0.4*0.75);
            \coordinate (c) at (2.1*2.5*0.75,1.5*0.4*0.75);
            \coordinate (d) at (2.25*2.5*0.75,5*0.4*0.75);
            \coordinate (e) at (2.8*2.5*0.75,10*0.4*0.75);
            \coordinate (f) at (3.1*2.5*0.75,4.8*0.4*0.75);
            \fill[red, opacity=0.3] (a) -- (b) -- (e) -- (f) -- (d) -- (c) -- (a) -- cycle;
        \end{tikzpicture}
        \caption{}\label{fig:equiv-c}
    \end{subfigure}
    \begin{subfigure}{0.24\linewidth}
        \centering
        \begin{tikzpicture}[scale=.75]
            \node[draw, circle, black, fill=blue, inner sep=0pt, minimum size=5pt, label={[xshift=-0.1cm, yshift=-0.6cm]$a'$}] (a) at (-1*2.5*0.75,6*0.4*0.75) {};
            \node[draw, circle, black, fill=blue, inner sep=0pt, minimum size=5pt, label={[xshift=-0.1cm, yshift=0.1cm]$b'$}] (b) at (-2*2.5*0.75,6.5*0.4*0.75) {};
            \node[draw, circle, black, fill=blue, inner sep=0pt, minimum size=5pt, label={[xshift=-0.2cm, yshift=-0.6cm]$c'$}] (c) at (-2.1*2.5*0.75,1.5*0.4*0.75) {};
            \node[draw, circle, black, fill=blue, inner sep=0pt, minimum size=5pt, label={[xshift=-0.2cm, yshift=-0.6cm]$d'$}] (d) at (-2.25*2.5*0.75,5*0.4*0.75) {};
            \node[draw, circle, black, fill=blue, inner sep=0pt, minimum size=5pt, label={[xshift=-0.3cm, yshift=-0.4cm]$e'$}] (e) at (-2.8*2.5*0.75,10*0.4*0.75) {};
            \node[draw, circle, black, fill=blue, inner sep=0pt, minimum size=5pt, label={[xshift=-0.3cm, yshift=-0.6cm]$f'$}] (f) at (-3.1*2.5*0.75,4.8*0.4*0.75) {};
            \coordinate (a) at (-1*2.5*0.75,6*0.4*0.75);
            \coordinate (b) at (-2*2.5*0.75,6.5*0.4*0.75);
            \coordinate (c) at (-2.1*2.5*0.75,1.5*0.4*0.75);
            \coordinate (d) at (-2.25*2.5*0.75,5*0.4*0.75);
            \coordinate (e) at (-2.8*2.5*0.75,10*0.4*0.75);
            \coordinate (f) at (-3.1*2.5*0.75,4.8*0.4*0.75);
            \fill[red, opacity=0.3] (a) -- (b) -- (e) -- (f) -- (d) -- (c) -- (a) -- cycle;
        \end{tikzpicture}
        \caption{}\label{fig:equiv-d}
    \end{subfigure}
  \caption{The pointsets depicted in \Cref{fig:equiv-a,fig:equiv-b} do not require the parity option to be shown $\sigma$-equivalent, since the bijection $f$ defined by $(a,b,c,d) \mapsto (b', d', c', a')$ already satisfies $\sigma(p_i, p_j, p_k) = \sigma(f(p_i), f(p_j), f(p_j))$ for every $\{p_i, p_j, p_k\} \subseteq \{a,b,c,d\}.$ On the other hand, no such bijection exists for~\Cref{fig:equiv-c,fig:equiv-d},  which require the parity option for $\sigma$-equivalence. }\label{fig:sigma-equiv}
  %We obtained~\Cref{fig:equiv-c,fig:equiv-d} computationally.
  \end{figure}


Third, the lex-order property breaks symmetry due to \emph{reflection}.
Reflecting a set of points~$S$ through a line (e.g., with the map $(x, y) \mapsto (-x, y)$)
preserves the presence of $k$-holes and convex $k$-gons.
Such reflected point sets are almost $\sigma$-equivalent to $S$,
but their orientations have all changed sign.
Consider the point sets in \Cref{fig:sigma-equiv}.
As a result,
we included the \emph{parity} flag in our Lean definition of $\sigma$-equivalence
to capture changes in orientations due to reflection,
where \lstinline|parity := true| changes sign of all orientations.
The lex-order property ensures that the point set is the lexicographically-larger reflection.

We prove that there always exists a $\sigma$-equivalent point set in canonical position.

%     While the first 3 conditions are now arguably standard in computational results regarding Erd\H{o}s-Szekeres type problems~\cite{scheucherTwoDisjoint5holes2020}, the last condition is a novelty introduced by Heule and Scheucher.
%     Interestingly, in the process of verifying the correctness of this symmetry-breaking assumption, we found a small error in the proof presented in~\cite{scheucherTwoDisjoint5holes2020} for the first $3$ conditions.
% The concrete theorem we prove is the following:
\begin{figure}
\begin{subfigure}{0.31\textwidth}
\centering
\begin{tikzpicture}
        \draw[help lines, color=gray!30, dashed] (-1.7,-1.3) grid (1.7,2.3);
    \draw[-latex, thick] (-1.8,0)--(1.8,0) node[right]{$x$};
    \draw[-latex, thick] (0,-1.4)--(0,2.4) node[above]{$y$};

    \draw[-, dashed, red] (-1.3, -1.4) -- (-1.3, 2.4);
    \draw[-, dashed, red] (-0.35, -1.4) -- (-0.35, 2.4);
    \draw[-, dashed, red] (0.6, -1.4) -- (0.6, 2.4);
    \node[draw, circle, fill=blue, text=white, inner sep=0pt, minimum size=5pt] (p1) at (-1.3, 0.25) {\scriptsize $1$};
    \node[draw, circle, fill=blue, text=white, inner sep=0pt, minimum size=5pt] (p2) at (-1.3, 0.9) {\scriptsize $2$};
    \node[draw, circle, fill=blue, text=white, inner sep=0pt, minimum size=5pt] (p3) at (-0.7, -0.45) {\scriptsize $3$};
    \node[draw, circle, fill=blue, text=white, inner sep=0pt, minimum size=5pt] (p4) at (-0.35, 0.35) {\scriptsize $4$};
    \node[draw, circle, fill=blue, text=white, inner sep=0pt, minimum size=5pt] (p5) at (-0.35, -0.8) {\scriptsize $5$};
    \node[draw, circle, fill=blue, text=white, inner sep=0pt, minimum size=5pt] (p6) at (0.6, 0.5) {\scriptsize $6$};
    \node[draw, circle, fill=blue, text=white, inner sep=0pt, minimum size=5pt] (p7) at (0.9, -0.6) {\scriptsize $7$};
    \node[draw, circle, fill=blue, text=white, inner sep=0pt, minimum size=5pt] (p8) at (1.4, 1.1) {\scriptsize $8$};
    \node[draw, circle, fill=blue, text=white, inner sep=0pt, minimum size=5pt] (p9) at (0.6, -1.1) {\scriptsize $9$};
\end{tikzpicture}
\caption{The original list of points.\\\phantom{x}\\\phantom{x}}\label{fig:symmetry-breaking-1}
\end{subfigure}
\hfil
\begin{subfigure}{0.31\textwidth}
    \begin{tikzpicture}
        \draw[help lines, color=gray!30, dashed] (-1.7,-1.3) grid (1.7,2.3);
    \draw[-latex, thick] (-1.8,0)--(1.8,0) node[right]{$x$};
    \draw[-latex, thick] (0,-1.4)--(0,2.4) node[above]{$y$};
    
        \node[draw, circle, fill=blue, text=white, inner sep=0pt, minimum size=5pt] (p1) at (-1.09601563215555, -0.7424619411597272) {\scriptsize $1$};
        \node[draw, circle, fill=blue, text=white, inner sep=0pt, minimum size=5pt] (p2) at (-1.5556349648261614, -0.28284245828783905) {\scriptsize $2$};
        \node[draw, circle, fill=blue, text=white, inner sep=0pt, minimum size=5pt] (p3) at (-0.17677682816699475, -0.8131727694796578) {\scriptsize $3$};
        \node[draw, circle, fill=blue, text=white, inner sep=0pt, minimum size=5pt] (p4) at (-0.49497474683057663, 8.087761060870946e-08) {\scriptsize $4$};
        \node[draw, circle, fill=blue, text=white, inner sep=0pt, minimum size=5pt] (p5) at (0.3181979186635819, -0.8131728503572684) {\scriptsize $5$};
        \node[draw, circle, fill=blue, text=white, inner sep=0pt, minimum size=5pt] (p6) at (0.07071080521204193, 0.7778174477512475) {\scriptsize $6$};
        \node[draw, circle, fill=blue, text=white, inner sep=0pt, minimum size=5pt] (p7) at (1.0606602064416402, 0.21213186104679588) {\scriptsize $7$};
        \node[draw, circle, fill=blue, text=white, inner sep=0pt, minimum size=5pt] (p8) at (0.21213232320457065, 1.767766918304512) {\scriptsize $8$};
        \node[draw, circle, fill=blue, text=white, inner sep=0pt, minimum size=5pt] (p9) at (1.202081470247393, -0.3535535870103234) {\scriptsize $9$};
    \end{tikzpicture}

\caption{All $x$-coordinates are different after rotating by $45^\circ$. We show such an angle always exists.}\label{fig:symmetry-breaking-2}
\end{subfigure}
\hfil
%
%\vspace{0.5cm}
%
\begin{subfigure}{0.31\textwidth}
    \begin{tikzpicture}
            \draw[help lines, color=gray!30, dashed] (-0.7,-1.3) grid (3.1,2.3);
    \draw[-latex, thick] (-0.8,0)--(3.2,0) node[right]{$x$};
    \draw[-latex, thick] (0,-1.4)--(0,2.4) node[above]{$y$};
    
    
%        \draw[help lines, color=gray!30, dashed] (-1,-0.9) grid (2.9,2.4);
%        \draw[-latex, thick] (-1,0)--(4,0) node[right]{$x$};
%        \draw[-latex, thick] (0,-1)--(0,2.5) node[above]{$y$};
    
        \node[draw, circle, fill=blue, text=white, inner sep=0pt, minimum size=5pt] (p1) at (0.4596193326706113, -0.45961948287188814) {\scriptsize $1$};
        \node[draw, circle, fill=blue, text=white, inner sep=0pt, minimum size=5pt] (p2) at (0.0, 0.0) {\scriptsize $2$};
        \node[draw, circle, fill=blue, text=white, inner sep=0pt, minimum size=5pt] (p3) at (1.3788581366591666, -0.5303303111918187) {\scriptsize $3$};
        \node[draw, circle, fill=blue, text=white, inner sep=0pt, minimum size=5pt] (p4) at (1.0606602179955846, 0.28284253916544966) {\scriptsize $4$};
        \node[draw, circle, fill=blue, text=white, inner sep=0pt, minimum size=5pt] (p5) at (1.8738328834897433, -0.5303303920694293) {\scriptsize $5$};
        \node[draw, circle, fill=blue, text=white, inner sep=0pt, minimum size=5pt] (p6) at (1.6263457700382034, 1.0606599060390867) {\scriptsize $6$};
        \node[draw, circle, fill=blue, text=white, inner sep=0pt, minimum size=5pt] (p7) at (2.6162951712678018, 0.4949743193346349) {\scriptsize $7$};
        \node[draw, circle, fill=blue, text=white, inner sep=0pt, minimum size=5pt] (p8) at (1.767767288030732, 2.0506093765923508) {\scriptsize $8$};
        \node[draw, circle, fill=blue, text=white, inner sep=0pt, minimum size=5pt] (p9) at (2.7577164350735544, -0.07071112872248436) {\scriptsize $9$};
    \end{tikzpicture}

\caption{After translating the leftmost point is at $(0,0)$.\\\phantom{x}}\label{fig:symmetry-breaking-3}
\end{subfigure}

\begin{subfigure}{0.31\textwidth}
    \begin{tikzpicture}
        \draw[help lines, color=gray!30, dashed] (-1.2,-0.9) grid (2.4,2.9);
        \draw[-latex, thick] (-1.2,0)--(2.5,0) node[right]{$x$};
        \draw[-latex, thick] (0,-1)--(0,3) node[above]{$y$};
    
        \node[draw, circle, fill=blue, text=white, inner sep=0pt, minimum size=5pt] (p1) at (-1.00000032679495, 2.1757135283877527) {\scriptsize $1$};
        \node[draw, circle, fill=green!60!black, text=white, inner sep=0pt, minimum size=5pt, label={[xshift=0.7cm, yshift=-0.4cm]$(0, \infty)$}] (p2) at (0, 3.85) {\scriptsize $2$};
        \node[draw, circle, fill=blue, text=white, inner sep=0pt, minimum size=5pt] (p3) at (-0.3846155721840648, 0.7252377698715973) {\scriptsize $3$};
        \node[draw, circle, fill=blue, text=white, inner sep=0pt, minimum size=5pt] (p4) at (0.2666664916498519, 0.9428090005013866) {\scriptsize $4$};
        \node[draw, circle, fill=blue, text=white, inner sep=0pt, minimum size=5pt] (p5) at (-0.2830190444100679, 0.5336655199142649) {\scriptsize $5$};
        \node[draw, circle, fill=blue, text=white, inner sep=0pt, minimum size=5pt] (p6) at (0.6521736801480853, 0.6148753963780468) {\scriptsize $6$};
        \node[draw, circle, fill=blue, text=white, inner sep=0pt, minimum size=5pt] (p7) at (0.1891890199433349, 0.3822198699068886) {\scriptsize $7$};
        \node[draw, circle, fill=blue, text=white, inner sep=0pt, minimum size=5pt] (p8) at (1.1599996167350177, 0.5656853177286622) {\scriptsize $8$};
        \node[draw, circle, fill=blue, text=white, inner sep=0pt, minimum size=5pt] (p9) at (-0.025641189145902285, 0.3626188636662086) {\scriptsize $9$};
       
    \end{tikzpicture}

\caption{Result after applying the map $(x, y) \mapsto (y/x, 1/x)$.}\label{fig:symmetry-breaking-4}
\end{subfigure}
\hfil
%
%\vspace{0.5cm}
%
\begin{subfigure}{0.31\textwidth}
    \begin{tikzpicture}
        \draw[help lines, color=gray!30, dashed] (-1.5,-0.9) grid (2.4,3.7);
        \draw[-latex, thick] (-1.4,0)--(2.5,0) node[right]{$x$};
        \draw[-latex, thick] (0,-1)--(0,3.8) node[above]{$y$};
    
           
        \node[draw, circle, fill=blue, text=white, inner sep=0pt, minimum size=5pt] (p1) at (-1.00000032679495, 2.1757135283877527) {\scriptsize $1$};
        \node[draw, circle, fill=blue, text=white, inner sep=0pt, minimum size=5pt] (p2) at (-1.2, 3.85) {\scriptsize $2$};
        \node[draw, circle, fill=blue, text=white, inner sep=0pt, minimum size=5pt] (p3) at (-0.3846155721840648, 0.7252377698715973) {\scriptsize $3$};
        \node[draw, circle, fill=blue, text=white, inner sep=0pt, minimum size=5pt] (p4) at (0.2666664916498519, 0.9428090005013866) {\scriptsize $4$};
        \node[draw, circle, fill=blue, text=white, inner sep=0pt, minimum size=5pt] (p5) at (-0.2830190444100679, 0.5336655199142649) {\scriptsize $5$};
        \node[draw, circle, fill=blue, text=white, inner sep=0pt, minimum size=5pt] (p6) at (0.6521736801480853, 0.6148753963780468) {\scriptsize $6$};
        \node[draw, circle, fill=blue, text=white, inner sep=0pt, minimum size=5pt] (p7) at (0.1891890199433349, 0.3822198699068886) {\scriptsize $7$};
        \node[draw, circle, fill=blue, text=white, inner sep=0pt, minimum size=5pt] (p8) at (1.1599996167350177, 0.5656853177286622) {\scriptsize $8$};
        \node[draw, circle, fill=blue, text=white, inner sep=0pt, minimum size=5pt] (p9) at (-0.025641189145902285, 0.3626188636662086) {\scriptsize $9$};
       
       
        
    \end{tikzpicture}

\caption{Point $2$ is brought back into the real plane.}\label{fig:symmetry-breaking-5}
\end{subfigure}
\hfil
\begin{subfigure}{0.31\textwidth}
    \begin{tikzpicture}
        \draw[help lines, color=gray!30, dashed] (-1.5,-0.9) grid (2.4,3.7);
        \draw[-latex, thick] (-1.4,0)--(2.5,0) node[right]{$x$};
        \draw[-latex, thick] (0,-1)--(0,3.8) node[above]{$y$};
    
           
        \node[draw, circle, fill=blue, text=white, inner sep=0pt, minimum size=5pt] (p1) at (-1.2, 3.85) {\scriptsize $1$};
        \node[draw, circle, fill=blue, text=white, inner sep=0pt, minimum size=5pt] (p2) at (-1.00000032679495, 2.1757135283877527) {\scriptsize $2$};
        \node[draw, circle, fill=blue, text=white, inner sep=0pt, minimum size=5pt] (p3) at (-0.3846155721840648, 0.7252377698715973) {\scriptsize $3$};
        \node[draw, circle, fill=blue, text=white, inner sep=0pt, minimum size=5pt] (p4) at (-0.2830190444100679, 0.5336655199142649) {\scriptsize $4$};
        \node[draw, circle, fill=blue, text=white, inner sep=0pt, minimum size=5pt] (p5) at (-0.025641189145902285, 0.3626188636662086) {\scriptsize $5$};
        \node[draw, circle, fill=blue, text=white, inner sep=0pt, minimum size=5pt] (p6) at (0.1891890199433349, 0.3822198699068886) {\scriptsize $6$};
        \node[draw, circle, fill=blue, text=white, inner sep=0pt, minimum size=5pt] (p7) at (0.2666664916498519, 0.9428090005013866) {\scriptsize $7$};
        \node[draw, circle, fill=blue, text=white, inner sep=0pt, minimum size=5pt] (p8) at (0.6521736801480853, 0.6148753963780468) {\scriptsize $8$};
        \node[draw, circle, fill=blue, text=white, inner sep=0pt, minimum size=5pt] (p9) at (1.1599996167350177, 0.5656853177286622) {\scriptsize $9$};
       
    \end{tikzpicture}

\caption{Points are relabeled from left to right.}\label{fig:symmetry-breaking-6}
\end{subfigure}



\caption{Illustration of the proof of the main symmetry breaking theorem. For simplicity we have ommited the illustration of the \emph{left-to-right} property. }\label{fig:symmetry-breaking}
\end{figure}
\begin{lstlisting}
theorem symmetry_breaking : ListInGenPos l →
  ∃ w : CanonicalPoints, Nonempty (l.toFinset ≃σ w.points.toFinset)
\end{lstlisting}


\begin{proof}[Proof Sketch]
The proof proceeds in 6 steps, illustrated in~\Cref{fig:symmetry-breaking}. In each of the steps, we will construct a new list of points that is $\sigma$-equivalent to the previous one, and the last one will be in canonical position.\footnote{Even though we defined $\sigma$-equivalence for sets of points, our formalization goes back and forth between sets and lists. Given that symmetry breaking distinguishes between the order of the points e.g., $x$-order, this proof proceeds over lists. All permutations of a list are immediately $\sigma$-equivalent.}
The main justification for each step is that, given that the function $\sigma$ is defined as a sign of the determinant, applying transformations that preserve (or, when \lstinline|parity := true|, uniformly reverse) the sign of the determinant will preserve (or uniformly reverse) the values of $\sigma$. In particular, given the identity $\det(AB) = \det(A)\det(B)$, if we apply a transformation to the points that corresponds to multiplying by a matrix $B$ such that $\det(B) > 0$, then $\sign(\det(A)) = \sign(\det(AB))$, and thus orientations will be preserved.
In step 1, we transform the list of points so that no two points share the same $x$-coordinate. This can be done by applying a rotation to the list of points, which corresponds to multiplying by a rotation matrix.
Rotations always have determinant $1$. 
In step 2, we translate all points by a constant vector $t$, by multiplying by a translation matrix, so that the left most point gets position $(0, 0)$, and naturally every other point will have a positive $x$-coordinate.
Let $L_2$ be the list of points after this transformation, excluding $(0,0)$ which we will denote by $p_1$.
Then, in step 3, we  apply the projective transformation $f: (x, y) \mapsto (y/x, 1/x)$ to every point in $L_2$, showing that this preserves orientations within $L_2$.
To see that this mapping is a $\sigma$-equivalence consider that 
\[
\begin{multlined}
 \sign \det \begin{pmatrix} p_x & q_x & r_x \\ p_y & q_y & r_y \\ 1 & 1 & 1 \end{pmatrix} =  \sign \det \left( \begin{pmatrix} 0 & 0 & 1 \\ 1 & 0 & 0\\ 0 & 1 & 0 \end{pmatrix}  \begin{pmatrix} \nicefrac{p_y}{p_x} & \nicefrac{q_y}{q_x} & \nicefrac{r_y}{r_x} \\ \nicefrac{1}{p_x} & \nicefrac{1}{q_x} & \nicefrac{1}{r_x} \\ 1 & 1 & 1 \end{pmatrix}  \begin{pmatrix} p_x & 0 & 0 \\ 0 & q_x & 0\\ 0 & 0 & r_x \end{pmatrix} \right)\\
                        = \sign \left(1 \cdot \det  \begin{pmatrix} \nicefrac{p_y}{p_x} & \nicefrac{q_y}{q_x} & \nicefrac{r_y}{r_x} \\ \nicefrac{1}{p_x} & \nicefrac{1}{q_x} & \nicefrac{1}{r_x} \\ 1 & 1 & 1 \end{pmatrix} \cdot  p_x q_x r_x  \right) = \sign \det \begin{pmatrix} p_y/p_x & q_y/q_x & r_y/r_x \\ 1/p_x & 1/q_x & 1/r_x \\ 1 & 1 & 1 \end{pmatrix}.
                        %  \tag{As $p_x q_x r_x > 0$ by step 2}
\end{multlined}
\]
To preserve orientations with respect to the leftmost point $(0, 0)$, we set $f( (0, 0)) = (0, \infty)$, a special point that is treated separately as follows.
As the function $\sigma$ takes points in $\mathbb{R}^2$ as arguments,
we need to define an extension
\(
  \sigma_{(0, \infty)}(q, r) = \begin{cases}
    1 & \text{if } q_x < r_x \\
    -1 & \text{otherwise}.  
  \end{cases},
\)
We then show that $\sigma((0, 0), q, r) = \sigma_{(0, \infty)}(f(q), f(r))$ for all points $q, r \in L_2$. 
In step 4, we sort the list $L_2$ by $x$-coordinate in increasing order, thus obtaining a list $L_3$.
This can be done while preserving $\sigma$-equivalence because sorting corresponds to a permutation, and all permutations of a list are $\sigma$-equivalent by definition.
In step 5, we check whether the \textsf{Lex order} condition above is satisfied in $L_3$, and if it is not, we reflect the pointset, which preserves $\sigma$-equivalence with \lstinline|parity := true|.
Note that in such a case we need to relabel the points from left to right again.
In step 6, we bring point $(0, \infty)$ back into the range by first finding a constant $c$ such that all points in $L_3$ are to the right of the line $y=c$, and then finding a large enough value $M$ such that $(c, M)$ has the same orientation with respect to the other points as $(0, \infty)$ did, meaning that 
\(\sigma((c, M), q, r) = \sigma_{(0, \infty)}(q, r)\) for every $q, r \in L_3$.
Finally, we note that the list of points obtained in step 6 satisfies the \text{CCW-order} property by the following reasoning:
if $1 < i < j \leq n$ are indices, then 
\begin{align*}
  \sigma(p_1, p_i, p_j) = 1 &\iff \sigma((c, M), p_i, p_j) = 1\\
                            &\iff \sigma_{(0, \infty)}(p_i, p_j) = 1\tag{By step 6}\\
                            &\iff (p_i)_x < (p_j)_x \tag{By definition of $\sigma_{(0, \infty)}$}\\
                            &\iff \textsf{true} \tag{By step 4, since points are sorted and $i < j$}.
\end{align*}
This concludes the proof.
\end{proof}




\section{The Empty Triangle Theorem}
\label{sec:empty-triangle}
Let us consider a much simpler theorem, which we call the \emph{empty triangle theorem}. Its proof will be helpful to motivate the different aspects of our formalization of the empty hexagon theorem.

\begin{theorem}[Empty Triangle Theorem]
  \label{thm:empty-triangle}
  Given a set $S$ of $n \geq 3$ points in general position, there exists $3$ points $a, b, c \in S$ such that no point $d \in S$ lies inside the triangle $abc$.
\end{theorem}

\begin{proof}[(Human Proof)]
    Let $a, b, x$ be any three points of $S$. 
    Define the relation $p \prec q$ to mean that the triangle $abp$ is contained inside the triangle $abq$. Note that $\prec$ is a finite partial order, and thus there must exist at least one minimal element for $\prec$. 
    Let $c$ be a minimal element of $\prec$, and now note that $abc$ cannot contain any other point $d \in  S$, as otherwise we would have $d \prec c$, contradicting the minimality of $c$. An illustration of this proof is presented in~\Cref{fig:empty-triangle}.
\end{proof}

\begin{figure}
\centering
\begin{tikzpicture}
\node[draw, circle, black, fill=black, inner sep=0pt, minimum size=5pt] (h) at (0,0) {};
\node[draw, circle, black, fill=black, inner sep=0pt, minimum size=5pt] (c) at (2,3) {};
\node[] at (2, 2.75) {$c$};
\node[draw, circle, black, fill=black, inner sep=0pt, minimum size=5pt] (p) at (1.2,4.75) {};
\node[draw, circle, black, fill=black, inner sep=0pt, minimum size=5pt] (d) at (4,1) {};
\node[draw, circle, black, fill=black, inner sep=0pt, minimum size=5pt] (e) at (5,6) {};
\node[draw, circle, black, fill=black, inner sep=0pt, minimum size=5pt] (f) at (2,7) {};
\node[draw, circle, black, fill=black, inner sep=0pt, minimum size=5pt] (g) at (4,5) {};
\node[draw, circle, black, fill=black, inner sep=0pt, minimum size=5pt] (a) at (-1,2) {};
\node[] at (-1.2, 1.75) {$a$};
\node[draw, circle, black, fill=black, inner sep=0pt, minimum size=5pt] (i) at (-2,3) {};
\node[draw, circle, black, fill=black, inner sep=0pt, minimum size=5pt] (j) at (-0.5, 3.5) {};
% \node[draw, circle, black, fill=black, inner sep=0pt, minimum size=5pt] (k) at (2.5,-1) {};
\node[draw, circle, black, fill=black, inner sep=0pt, minimum size=5pt] (l) at (3,6) {};
\node[draw, circle, black, fill=black, inner sep=0pt, minimum size=5pt] (m) at (-4,2) {};
% \node[draw, circle, black, fill=black, inner sep=0pt, minimum size=5pt] (n) at (-2, -2) {};
\node[draw, circle, black, fill=black, inner sep=0pt, minimum size=5pt] (o) at (2, 0) {};

\node[draw, circle, black, fill=black, inner sep=0pt, minimum size=5pt] (b) at (5,1) {};
\node[] at (5.2, 0.75) {$b$};

\node[draw, circle, black, fill=black, inner sep=0pt, minimum size=5pt] (q) at (3.2, 3.75) {};

\draw[ultra thick, dashed, blue] (a) -- (b);
\draw[fill=green, opacity=0.5] (a.center) -- (b.center) -- (c.center) -- cycle;
\draw[dashed] (a.center) -- (b.center) -- (l.center) -- cycle;
\draw[dashed] (a.center) -- (b.center) -- (g.center) -- cycle;
\draw[dashed] (a.center) -- (b.center) -- (e.center) -- cycle;
\draw[dashed] (a.center) -- (b.center) -- (q.center) -- cycle;
\draw[dashed] (a.center) -- (b.center) -- (f.center) -- cycle;
\draw[dashed] (a.center) -- (b.center) -- (p.center) -- cycle;

\end{tikzpicture}
\caption{An illustration of the proof for~\Cref{thm:empty-triangle}.}
\label{fig:empty-triangle}
\end{figure}

Instead of formalizing the above proof, we will formalize a SAT-based proof, with the goal of approaching the formalization of the empty hexagon theorem.
Naturally, to use a  SAT-solver we need to parameterize~\Cref{thm:empty-triangle} by $n = |S|$, the number of points, and generate a different proof for each value of $n \geq 3$. For example, consider the following theorem.

\begin{lstlisting}
  theorem EmptyTriangle10Theorem (pts : Finset Point)
    (gp : PointFinsetInGeneralPosition pts)
    (h : pts.card = 10) :
    ∃ (p q r : Point), {p, q, r} ⊆ pts ∧ EmptyTriangleIn p q r pts
\end{lstlisting}

To complete the statement of the theorem, we must define~\texttt{EmptyTriangleIn}, and before that what it means for a point to be \emph{inside} a triangle.

\begin{lstlisting}
  def pt_in_triangle (a : Point) (p q r : Point) : Prop :=
    ∃ p' q' r', ({p', q', r'} : Set Point) = {p, q, r} ∧
      Sorted₃ p' q' r' ∧
      Sorted₃ p' a r' ∧
      a ≠ q' ∧ -- this isn't needed if p,q,r are in GP
      σ p' q' r' = σ p' a r' ∧
      σ p' a q' = σ p' r' q' ∧
      σ q' a r' = σ q' p' r'
  
  /-- S is an empty triangle relative to pts -/
  structure EmptyTriangleIn (p q r : Point) (pts : Finset Point) : Prop :=
    gp : InGeneralPosition₃ p q r
    empty: ∀ a ∈ pts, ¬(pt_in_triangle a p q r)
\end{lstlisting}



%------------------------------------------------------------------------------
% Index
%\printindex

%------------------------------------------------------------------------------
\end{document}

