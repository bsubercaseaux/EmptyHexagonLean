Our formalization is closely related to a prior development
in which Marić put proofs of $g(6) \leq 17$ on a more solid foundation \cite{19maric_fast_formal_proof_erdos_szekeres_conjecture_convex_polygons_most_six_points}.
The inequality,
originally obtained by Szekeres and Peters \cite{06szekeres_computer_solution_17_point_erdos_szekeres_problem}
using a specialized, unverified search algorithm,
was confirmed by Marić using a formally verified SAT encoding.
Marić introduced an optimized encoding based on nested convex hull structures,
which when combined with performance advances in modern SAT solvers,
improved the search time significantly over the unverified computation.

Our work focuses on the closely related problem
of determining $k$-hole numbers $h(k)$.
Rather than devising a new SAT encoding,
we use essentially the same encoding presented in Heule and Scheucher~\cite{emptyHexagonNumber}.
Interestingly,
a (verified) proof of $g(6) \leq 17$ can be obtained
as a corollary of our development.
We can assert the hole variables $\hvar_{a,b,c}$ as true rather than false,
while leaving the remainder of the encoding in~\Cref{fig:full-encoding} unchanged,
which trivializes constraints about emptiness
so that only convexity constraints remain.\footnote{
This modification was performed by an author
who did not understand this part of the proof,
nevertheless having full confidence in its correctness
thanks to the Lean kernel having checked every assertion.}
The resulting CNF formula
asserts the existence of a set of $n$ points
with no convex $6$-gon.
We checked this formula to be unsatisfiable for $n = 17$,
giving the same result as Marić:
\begin{lstlisting}
theorem gon_6_theorem (pts : List Point) (gp : ListInGenPos pts)
    (h : pts.length ≥ 17) : HasConvexKGon 6 pts.toFinset
\end{lstlisting}

Using this result,
we can perform a direct comparison against Marić's encoding.
On a personal laptop,
we find that it takes negligible time (below 1s)
for our verified Lean encoder to output the CNF.
In contrast,
Marić's encoder, extracted from Isabelle/HOL code,
took 437s to output a CNF
(this was compiled on Isabelle/HOL 2016,
the latest version that accepts the codebase without broader changes).
To circumvent the encoder slowness,
Marić wrote a C++ encoder
whose code was manually compared against the Isabelle/HOL specification,
a step which we do not have to take.

As for the encodings,
ours took 28s to solve,
while the Marić encoding took 787s (both using \textsf{cadical}).
This difference is likely accounted for in the relative size of the encodings,
in particular their symmetry breaking strategies.
For $k=6$ and $n$ points,
the encoding of Heule and Scheucher uses $O(n^4)$ clauses,
whereas the one of Marić uses $O(n^6)$ clauses.
They are based on different ideas:
the former as detailed in~\Cref{sec:symmetry-breaking},
whereas the latter on nested convex hulls.
The different approaches have been discussed by Scheucher \cite{scheucherTwoDisjoint5holes2020}.
This progress in solve times
represents an encouraging state of affairs;
we are optimistic that if continued,
it could lead to an eventual resolution of $g(7)$.

Further differences regard what exactly was formally proven.
As with most work in this area,
we use the combinatorial abstraction of triple orientations.
We and Marić alike show that point sets in $\mathbb R^2$
satisfy orientation properties (\Cref{sec:triple-orientations}).
However, our work goes further in building the connection
between geometry and combinatorics:
our definitions of convexity and emptiness (\Cref{sec:geometry}),
and consequently the theorem statements,
are geometric ones based on convex hulls
as defined in Lean's \texttt{mathlib}~\cite{The_mathlib_Community_2020}.
In contrast, Marić axiomatizes these properties in terms of $\sigma$.
A skeptical reviewer must manually verify that these combinatorial definitions
correspond to the desired geometric concept.

A final point of difference concerns the verification of SAT proofs.
Marić fully reconstructs some of their results,
though not the main one for $g(6)$,
in an NbE-based proof checker for Isabelle/HOL.
We make no such attempt for the time being,
instead passing our SAT proofs through the
formally verified proof checker \texttt{cake\_lpr} \cite{tanVerifiedPropagationRedundancy2023}
and asserting unsatisfiability of the CNF as an axiom in Lean.
This limitation is discussed in the concluding remarks.
