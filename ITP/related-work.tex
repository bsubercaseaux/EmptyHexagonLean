Our formalization is closely related to a prior development
in which Marić put proofs of $g(6) \leq 17$ on a more solid foundation \cite{19maric_fast_formal_proof_erdos_szekeres_conjecture_convex_polygons_most_six_points}.
The inequality,
originally obtained by Szekeres and Peters \cite{06szekeres_computer_solution_17_point_erdos_szekeres_problem}
using a bespoke, unverified search algorithm,
was confirmed in the work of Marić
by means of a formally verified SAT encoding.
Introducing an optimized encoding based on nested convex hull structures
and leveraging accumulated progress in SAT solving technology, % BS: I think it's fine without citation
this approach also improved on the time to solution

Our work focuses on the closely related problem
of determining $k$-hole numbers $h(k)$.
Rather than devising a new SAT encoding,
we use one from Heule and Scheucher~\cite{emptyHexagonNumber}
in a lightly modified form.
Interestingly,
a proof of $g(6) \leq 17$ can be obtained easily
as a corollary of our development:
asserting the hole variables $\hvar_{a,b,c}$ as unit clauses,
leaving the remainder of the encoding in~\Cref{fig:full-encoding} unchanged,
trivializes constraints that have to do with emptiness
so that only convexity constraints remain.\footnote{
This modification was performed by an author
who did not understand this part of the proof,
nevertheless having full confidence in its correctness
thanks to the Lean kernel having checked every assertion.}
The resulting CNF formula
asserts the existence of a set of $n$ points
with no convex $6$-gon.
We find this formula to be unsatisfiable for $n = 17$,
giving the classic
\begin{lstlisting}
theorem gon_6_theorem (pts : List Point) (gp : ListInGenPos pts)
    (h : pts.length ≥ 17) : HasConvexKGon 6 pts.toFinset
\end{lstlisting}
% However, it would not be as easy
% to attack $g(k)$ for $k \neq 6$ using this particular encoding. BS: just to save space; it's not a bad sentence but also not key

As both formalizations are executable,
we can perform a direct comparison against Marić's encoding.
On a personal laptop,
we find that it takes negligible time (below 1s)
for our verified Lean encoder to generate a CNF,
which then takes around 28s to solve.
In contrast,
Marić's encoder, running in Isabelle/HOL 2016,
% \footnote{
% The latest version that accepts the codebase
% without broader changes.} BS: just to save space
takes 437s to generate a CNF,
which then takes around 787s to solve.
To avoid encoding overhead,
Marić resorts to trusting a C++ encoder
whose code has been manually compared against the Isabelle/HOL specification:
something we do not have to do.
The difference in solving time
can likely be accounted for by differences in the respective encodings,
in particular their symmetry breaking strategies.
The encoding of Heule and Scheucher uses $O(n^4)$ clauses
to look for a $6$-hole or $6$-gon in $n$ points,
whereas the one of Marić uses $O(n^6)$ clauses.
They are based on different ideas:
the former as detailed in~\Cref{sec:symmetry-breaking},
whereas the latter on nested convex hulls.
The different approaches have been discussed by Scheucher \cite{scheucherTwoDisjoint5holes2020}.
The progress in solving times
represents an encouraging state of affairs;
we are optimistic that if continued,
it could lead to an eventual resolution of $g(7)$.

Further differences regard what exactly was formally proven.
Like most work in this area,
we use the combinatorial abstraction of triple orientations.
We and Marić alike show that point sets in $\mathbb R^2$
satisfy orientation properties (\Cref{sec:triple-orientations}).
However, our work goes further in building the connection
between geometry and combinatorics:
our definitions of convexity and emptiness (\Cref{sec:geometry}),
and consequently the theorem statements,
are the geometric ones based on convex hulls
as defined in Lean's \texttt{mathlib}~\cite{The_mathlib_Community_2020}.
In contrast, Marić axiomatizes these as being about specific values of $\sigma$.
Thus one has to trust that each combinatorial abstraction
corresponds exactly to the geometric concept.

A final point of difference concerns the verification of SAT proofs.
Marić fully reconstructs some of their results,
though not the main one for $g(6)$,
in an NbE-based proof checker for Isabelle/HOL.
We make no such attempt for the time being,
instead passing our SAT proofs through the the formally verified proof checker \texttt{cake\_lpr} \cite{tanVerifiedPropagationRedundancy2023}
and asserting unsatisfiability of the CNF as an axiom in Lean.
Thus we are trusting that the CNF formula produced by the verified Lean encoder
is the same one whose unsatisfiability was checked by \texttt{cake\_lpr}.
Everything else has been verified.

% We know of no other formal proofs of Erd\H{o}s-Szekeres-type problems.

% WN: Maybe put this in the symmetry breaking section?


% WN: Anything else that should be mentioned here?
