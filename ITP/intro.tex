Mathematicians are often rightfully skeptical of proofs that rely on extensive computation (e.g., the controversy around the four color theorem~\cite{Walters2004ItAT}).
Nonetheless, many mathematically-interesting theorems have been resolved that way.
SAT solving in particular has been a powerful tool for mathematics, successfully resolving
Keller's conjecture~\cite{brakensiek2023resolution},
the packing chromatic number of the infinite grid~\cite{Subercaseaux_Heule_2023},
the Pythagorean triples problem~\cite{Heule_2016},
Lam's problem~\cite{21bright_sat_based_resolution_lams_problem},
and one case of the Erd\H{o}s discrepancy conjecture~\cite{konev2014sat}.
All of these proofs rely on the same two-step structure:
\begin{itemize}
\item \textbf{(Reduction)} Show that the mathematical theorem of interest is true if a concrete propositional formula~$F$ is unsatisfiable.
\item \textbf{(Solving)} Show that $F$ is indeed unsatisfiable.
\end{itemize}

% \footnote{A variant of this procedure uses \emph{satisfying assignments} for $P$ to construct explicit witnesses for the original theorem, but we focus on the unsatisfiable case.}
% \end{itemize}

Formal methods researchers have devoted significant attention to making the \emph{solving} step reliable, reproducible and trustworthy.
Modern SAT solvers produce proofs of unsatisfiability in formal systems
such as DRAT~\cite{drat-trim14}
that can in turn be checked with verified proof checkers
such as \texttt{cake\_lpr}~\cite{tanVerifiedPropagationRedundancy2023}. 
These tools ensure that when a SAT solver declares a formula~$F$ to be unsatisfiable, the formula is indeed unsatisfiable.
In contrast, the \emph{reduction} step can use problem-specific mathematical insights that, when left unverified, threaten the trustworthiness of SAT-based proofs in mathematics. 
% The correctness of the \emph{reduction} step, however, has received 
% leaving room for doubt that any results relying on such reduction arguments are correct.
A perfect example of the complexity of this reduction step can be found in a recent breakthrough of Heule and Scheucher~\cite{emptyHexagonNumber} in discrete computational geometry. 
They constructed (and solved) a formula $F$ whose unsatisfiability implies that every set of 30 points, without three in a common line, must contain an empty convex hexagon.
However, as is common with such results, their reduction argument was only sketched, relied heavily on intuition,
and left several gaps to be filled in.
 

% result from Heule and Scheucher~\cite{emptyHexagonNumber} falls into this category.
% They resolve a variant of the Happy Ending Problem,
% in particular that every set of 30 points in general position contains an empty convex hexagon.
% Their proof relies on a complicated reduction to SAT
% involving numerous nontrivial geometric optimizations and symmetry-breaking arguments.

In this paper we complete and formalize the reduction of Heule and Scheucher in the Lean theorem prover~\cite{demouraLeanTheoremProver2015}. We do so by connecting existing geometric definitions
in the mathematical proof library \texttt{mathlib}~\cite{The_mathlib_Community_2020}
to the unsatisfiability of a particular SAT instance, thus setting a new standard for verifying results which rely on extensive computation.
Our formalization is publicly available at \url{https://github.com/bsubercaseaux/EmptyHexagonLean}.


\subparagraph*{Verification of SAT proofs.}
Formal verification plays a crucial role in certifying the \emph{solving} step of SAT-based results.
For example, theorem provers and formal methods tools have been used to verify solvers~\cite{10maric_formal_verification_modern_sat_solver_shallow_embedding_isabelle_hol,oeVersatVerifiedModern2012,skotam_creusat_2022}
and proof checkers~\cite{lammichEfficientVerifiedSAT2020,tanVerifiedPropagationRedundancy2023}.
However, the \emph{reduction} step has not received similar scrutiny.
Some work has been done to verify the reductions to SAT underlying these kinds of mathematical results.
The solution to the Pythagorean triples problem
was verified in the \textsf{Coq} proof assistant
by Cruz-Filipe and coauthors~\cite{formalPythagoreanTriples,LPAR-21:Formally_Proving_Boolean_Pythagorean}.
More generally,
Giljeg\r{a}rd and Wennerbreck~\cite{GilAndWennerbeck} provide a \textsf{CakeML} library
of verified SAT encodings,
which they used to write verified reductions from different puzzles
(e.g., Sudoku, Kakuro, the \emph{N-queens} problem).
The reduction verification techniques we use in this paper
are based on that of Codel, Avigad, and Heule~\cite{Cayden} in the Lean theorem prover.

Formal verification for SAT-based combinatorial geometry
was pioneered by Marić~\cite{19maric_fast_formal_proof_erdos_szekeres_conjecture_convex_polygons_most_six_points}.
He developed a reduction of a case of the Happy Ending Problem to SAT
and formally verified it in \textsf{Isabelle/HOL}.
We give a detailed comparison between his work and ours in~\Cref{sec:related-work}.

\subparagraph*{Lean.}
Initially developed by Leonardo de Moura in 2013~\cite{demouraLeanTheoremProver2015},
the Lean theorem prover has become a popular choice for formalizing modern mathematical research.
Recent successes include the~\emph{Liquid Tensor Experiment}~\cite{Castelvecchi2021}
and the proof of the polynomial Freiman-Ruzsa conjecture~\cite{gowers2023conjecture, slomanATeamMathProves2023},
both of which brought significant attention to Lean.
A major selling point for Lean is the \textsf{mathlib} project~\cite{The_mathlib_Community_2020},
a monolithic formalization of foundational mathematics.
By relying on \textsf{mathlib} for definitions, lemmas, and proof tactics,
mathematicians can focus on the interesting components of a formalization
while avoiding duplication of proof efforts across formalizations.
In turn, by making a formalization compatible with \textsf{mathlib},
future proof efforts can rely on work done today.
In this spirit, we connect our results to~\textsf{mathlib} as much as possible.

\subparagraph*{The Empty Hexagon Number.}
In the 1930s,
Erd\H{o}s and Szekeres, inspired by Esther Klein, showed that for any $k \geq 3$,
one can find a sufficiently large number $n$
such that every $n$ points in \emph{general position}
(i.e., with no three points collinear)
contain a convex \emph{$k$-gon}, i.e., a convex polygon with $k$ vertices~\cite{35erdos_combinatorial_problem_geometry}.
The minimal such $n$ is denoted $g(k)$.
The same authors later showed that $g(k) > 2^{k-2}$
and conjectured that this bound is tight~\cite{60erdos_some_extremum_problems_elementary_geometry}.
Indeed, it is known that $g(5) = 9$ and $g(6) = 17$,
with the latter result obtained by Szekeres and Peters 71 years after the initial conjecture
via exhaustive computer search~\cite{06szekeres_computer_solution_17_point_erdos_szekeres_problem}.
Larger cases remain open,
with $g(k) \leq 2^{k+o(k)}$ the best known upper bound~\cite{suk2017erdos,holmsen2017two}.
This problem is now known as the \emph{Happy Ending Problem},
as it led to the marriage of Klein and Szekeres.

In a similar spirit,
Erd\H{o}s defined $h(k)$
to be the minimal number of points in general position
that is guaranteed to contain a \emph{$k$-hole},
or \emph{empty $k$-gon},
meaning a convex $k$-gon with no other point inside.
It is easy to check that $h(3) = 3$ and $h(4) = 5$.
In 1978, Harborth established that $h(5) = 10$~\cite{Harborth1978}.
Surprisingly, in 1983, Horton discovered constructions of arbitrarily large point sets that 
avoid $k$-holes for $k \geq 7$~\cite{hortonSetsNoEmpty1983}.
Only $h (6)$ remained.
The \emph{Empty Hexagon Theorem},
establishing $h(6)$ to be finite,
was proven independently by Gerken and Nicolás in 2006~\cite{gerkenEmptyConvexHexagons2008,nicolasEmptyHexagonTheorem2007}.
In 2008, Valtr narrowed the range of possible values down to $30 \leq h(6) \leq 1717$,
where the problem remained until the breakthrough by Heule and Scheucher~\cite{emptyHexagonNumber},
who used a SAT solver to prove that $h(6) \leq 30$,
a result we refer to as the \emph{Empty Hexagon Number}.
