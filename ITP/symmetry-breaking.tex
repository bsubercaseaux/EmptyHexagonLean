\emph{Symmetry breaking} plays a key role in modern SAT-solving by substantially reducing the search space of satisfying assignments to a formula~\cite{biereHandbookSatisfiabilityVolume2009}. For example, if one knows all satisfying assignments to a formula $F$ have either (i) $x_1 = 0, x_2 = 1$, or  (ii) $x_1 = 1, x_2 = 0$, and there is a bijection between satisfying assignments of form (i) and satisfying assignments of form (ii), then one can assume, \emph{without loss of generality}, that $x_1 = 0, x_2 = 1$, and thus add unit clauses $\ov{x_1}$ and $x_2$ to the formula $F$ while preserving its satisfiability. 

In the context of the Empty Hexagon Number, the symmetry breaking done by Heule and Scheucher consist in assuming, \emph{without loss of generality}, that in order to search for a list of $30$ points in general position without a $6$-hole, one can search only amongst lists of $30$ points in \emph{canonical} position, which are defined as follows.   
\begin{definition}[Canonical Position]
A list $L = (p_1,\ldots, p_{n})$ of points is said to be in canonical position if it satisfies all the following properties:
\begin{itemize}
    \item \textbf{($x$-order)} The points are sorted with respect to their $x$-coordinates, i.e., $x(p_i) < x(p_j)$ for all $1 \leq i < j \leq n$.
    \item \textbf{(General Position)} No three points are collinear, i.e., for all $1 \leq i < j < k \leq n$, we have $\sigma(p_i, p_j, p_k) \neq 0$. 
    \item \textbf{(CCW-order,~\Cref{eq:constraint-8})} All orientations $\sigma(p_0, p_i, p_j)$, with $1 \leq i < j \leq n$, are counterclockwise.
    \item \textbf{(Left-Right order,~\Cref{eq:constraint-9})} The list of orientations $\left(\sigma\left(p_{\lceil \frac{n}{2} \rceil -1}, p_{\lceil \frac{n}{2} \rceil},p_{\lceil \frac{n}{2} \rceil+1}\right), \ldots, \sigma\left(p_2, p_3, p_4\right) \right)$ is lexicographically smaller than the list $\left(\sigma\left(p_{\lfloor \frac{n}{2} \rfloor  + 1}, p_{\lfloor \frac{n}{2} \rfloor+2},p_{\lfloor \frac{n}{2} \rfloor+3}\right), \ldots, \sigma\left(p_{n-2}, p_{n-1}, p_{n}\right) \right)$.
\end{itemize}
\end{definition}
    While the first 3 conditions are now arguably standard in computational results regarding Erd\H{o}s-Szekeres type problems~\cite{scheucherTwoDisjoint5holes2020}, the last condition is a novelty introduced by Heule and Scheucher.
    Interestingly, in the process of verifying the correctness of this symmetry-breaking assumption, we found a small error in the proof presented in~\cite{scheucherTwoDisjoint5holes2020} for the first $3$ conditions. 
The concrete theorem we prove is the following:

\begin{lstlisting}
  theorem symmetry_breaking (l : GenPosPointList):
  ∃ w : CanonicalPointList (l =σ w.points)
\end{lstlisting}
  

\begin{proof}[Proof Sketch]
\todo[inline]{...}
\end{proof}


\begin{lstlisting}
....

\end{lstlisting}