\begin{figure}
  \caption{Encoding based on that of Heule and Scheucher for the Empty Hexagon Number~\cite{emptyHexagonNumber}. Each line determines a set of clauses. Unsatisfiability of the formula below for $n=30$ implies $h(6) \leq 30$, as detailed throughout the paper.}
  \label{fig:full-encoding}
% \begin{framed}
  \begin{spreadlines}{16pt}
\begin{gather}
\hfsetfillcolor{green!10}
\hfsetbordercolor{green!60!black}
\tikzmarkin{b}(12.0,-0.9)(-0.5,0.5)
  \cvar_{i; a,b, c} \rightarrow \left(\left(\orvar_{a,b,c} \leftrightarrow \orvar_{a, i, c}  \right) \land \left(\orvar_{a,b,c} \leftrightarrow \ov{\orvar_{a, i, b}}  \right)\right) \text{ for all } 2 \leq a < i < b < c \leq n\label{eq:insideClauses1}\\
  \cvar_{i; a,b, c} \rightarrow \left(\left(\orvar_{a,b,c} \leftrightarrow \orvar_{a, i, c}  \right) \land \left(\orvar_{a,b,c} \leftrightarrow \ov{\orvar_{b, i, c}}  \right)\right) \text{ for all } 2 \leq a < b < i < c \leq n\label{eq:insideClauses2}\\
  \tikzmarkend{b}\Big(\bigwedge_{\substack{a < i < c\\ i \neq b}} \ov{\cvar_{i; a,b,c}}\Big) \rightarrow \hvar_{a, b, c} \quad \text{ for all } 2 \leq a < b < c \leq n\label{eq:holeDefClauses1}\\
%
\tikzmarkin{a}(12.0,-0.3)(-0.5,0.5)
  \orvar_{a, b, c} \land \orvar_{a, c, d} \rightarrow \orvar_{a, b, d} \quad \text{ for all } 2 \leq a < b < c < d \leq n\label{eq:signotopeClauses11}\\
  \tikzmarkend{a}\ov{\orvar_{a, b, c}} \land \ov{\orvar_{a, c, d}} \rightarrow \ov{\orvar_{a, b, d}} \quad \text{ for all } 2 \leq a < b < c < d \leq n \label{eq:signotopeClauses12}\\
%
\hfsetfillcolor{blue!10}
\hfsetbordercolor{blue!60!black}
\tikzmarkin{c}(12.0,-0.4)(-0.5,0.6)
  % \orvar_{1, b, c} \quad \text{ for all } 2 \leq b < c \leq n \label{eq:revLexClauses}\\
  \tikzmarkend{c}\left(\orvar_{\lceil \frac{n}{2} \rceil -1, \lceil \frac{n}{2} \rceil,\lceil \frac{n}{2} \rceil+1}, \ldots, \orvar_{2,3,4} \right) \succeq_{\text{lex}} \left(\orvar_{\lfloor \frac{n}{2}\rfloor +1,  \lfloor \frac{n}{2}\rfloor +2, \lfloor \frac{n}{2}\rfloor +3}, \ldots, \orvar_{n-2, n-1, n} \right)\label{eq:revLexClauses}\\
%
\hfsetfillcolor{orange!10}
\hfsetbordercolor{orange!60!black}
\tikzmarkin{d}(12.0,-0.3)(-0.5,0.5)
  \ov{\orvar_{a,b,c}} \land \ov{\orvar_{b,c,d}} \rightarrow \uvar_{a, c, d} \quad \text{ for all } 2 \leq a < b < c < d \leq n\label{eq:capDef}\\
  \orvar_{a, b, c} \land \orvar_{b, c, d} \rightarrow \vvar_{a, c, d} \quad \text{ for all } 2 \leq a < b < c < d \leq n \label{eq:cupDef}\\
  \uvar_{a,b,c} \land \ov{\orvar_{b,c,d}} \land \hvar_{a,b,d} \rightarrow \ufvar_{a, c, d} \quad \text{ for all } 2 \leq a < b < c < d \leq n,\; a+1<b\label{eq:capFDef}\\
  \uvar_{a, c, d} \rightarrow \ov{\orvar_{a,c,d}} \quad \text{ for all } 2 \leq a < c < d \leq n,\ a+1<c\label{eq:capDef2}\\
  \tikzmarkend{d}\vvar_{a, c, d} \rightarrow \orvar_{a,c,d} \quad \text{ for all } 2 \leq a < c < d \leq n,\; a+1<c\label{eq:cupDef2}\\
%
\hfsetfillcolor{red!10}
\hfsetbordercolor{red!60!black}
\tikzmarkin{e}(12.0,-0.5)(-0.5,0.5)
  \neg(\ufvar_{a,d,e} \land \orvar_{a, b, e}) \quad \text { for all } 2 \leq a < d < e \leq n, \; a < b < e, \; a+2 < d\label{eq:no6Hole1Below}\\
  \neg(\ufvar_{a,d,e} \land \ov{\orvar_{d, e, f}}) \quad \text { for all } 2 \leq a < d < e < f\leq n, \; a+2 < d\label{eq:no6Hole4Above}\\
  \neg(\uvar_{a,c,d} \land \vvar_{a, c', d} \land \hvar_{a,c,c'}) \quad \text{ for all } 2 \leq a < c < c' < d \leq n, \; a+1 < c\label{eq:no6Hole2Below1}\\
  \neg(\uvar_{a,c,d} \land \vvar_{a, c', d} \land \hvar_{a,c',c}) \quad \text{ for all } 2 \leq a < c' < c < d \leq n, \; a+1 < c'\label{eq:no6Hole2Below2}\\
  \tikzmarkend{e}\neg(\vvar_{a,c,d} \land \orvar_{c, d, e} \land \hvar_{a,c,e}) \quad \text{ for all } 2 \leq a < c < d < e \leq n, \; a+1 < c\label{eq:no6Hole3Below}
  \end{gather}
\end{spreadlines}
% \end{framed}
\end{figure}


\subsection{Base variables}
Ranging from the initial work of Peters and Szekeres~\cite{szekeres_peters_2006} to the recent work of Heule and Scheucher~\cite{emptyHexagonNumber}, the base variables are defined as follows. Assuming a list of points $L = (p_1, \ldots, p_n)$,
we construct boolean variables $\orvar_{a, b, c}$ that represent $\sigma(p, q, r) = 1$, provided $L[a] = p, L[b] = q, L[c] =r$.\footnote{Given that pointsets are assumed to be in general position we have $\neg \orvar_{i,j,k} \iff \sigma(p, q, r) = -1$.} If one were to create a variable $\orvar_{a,b,c}$ for every triple of points $p \neq q \neq r$, that would amount to $n(n-1)(n-2)$ variables for $n$ points. However, the orientations of triples $(p, q, r)$ and $(q, r, p)$ or $(r, q, p)$ contain redundant information: if $p,q,r$ are oriented counterclockwise, then $q,r,p$ and $r,p,q$ are also oriented counterclockwise, whereas $p,r,q$ is oriented clockwise. This is captured by the following two fundamental asymmetries:
\begin{lstlisting}
  lemma σ_perm₁ (p q r : Point) : σ p q r = -σ q p r
  lemma σ_perm₂ (p q r : Point) : σ p q r = -σ p r q
\end{lstlisting}
As a result, it suffices to have variables $\orvar(a, b, c)$ with $1 \leq a < b < c \leq n$, thus reducing the number of variables by a factor of $3! = 6$.
Next, variables $\cvar_{i;a,b,c}$ correspond to whether point $i$ is inside the triangle $abc$, with $a < b < c$.
If we assume that points will be sorted from left to right (which will be justified in~\Cref{sec:symmetry-breaking}), then we only need to consider $a < i < c$,
and by combining the definition of \lstinline|σPtInTriangle| with the \lstinline|σ_perm| lemmas, we obtain precisely the definition of $\cvar_{i;a,b,c}$ of~\Cref{eq:insideClauses1,eq:insideClauses2}.
Finally, the variables $\hvar_{a, b, c}$ correspond to whether the triangle $abc$ is empty, and are defined in terms of $\cvar_{i;a,b,c}$ as in~\Cref{eq:holeDefClauses1}.
