Having established the reduction to orientations,
and the symmetry-breaking assumption of canonicity,
we now turn to the construction of a CNF formula $\phi_n$
whose unsatisfiability would imply
that every set of $n$ points
contains a $6$-hole.\footnote{
  Satisfiability of $\phi_n$ would \emph{not} necessarily imply
  the existence of a point set without a $6$-hole.
  First, the mapping from sets of points to assignments of triple orientations is not surjective.
  Second, even if it were, $\phi_n$ is tailored to looking for an unsatisfiability proof,
  using implication rather than bi-implication in some of the variable-defining clauses.
}
The formula is detailed in~\Cref{fig:full-encoding}.

\begin{figure}
  \label{fig:full-encoding}
% \begin{framed}
  \begin{spreadlines}{16pt}
\begin{gather}
\hfsetfillcolor{green!10}
\hfsetbordercolor{green!60!black}
\tikzmarkin{b}(12.0,-0.9)(-0.5,0.5)
  \cvar_{i; a,b, c} \rightarrow \left(\left(\orvar_{a,b,c} \leftrightarrow \orvar_{a, i, c}  \right) \land \left(\orvar_{a,b,c} \leftrightarrow \ov{\orvar_{a, i, b}}  \right)\right) \text{ for all } 2 \leq a < i < b < c \leq n\label{eq:insideClauses1}\\
  \cvar_{i; a,b, c} \rightarrow \left(\left(\orvar_{a,b,c} \leftrightarrow \orvar_{a, i, c}  \right) \land \left(\orvar_{a,b,c} \leftrightarrow \ov{\orvar_{b, i, c}}  \right)\right) \text{ for all } 2 \leq a < b < i < c \leq n\label{eq:insideClauses2}\\
  \tikzmarkend{b}\Big(\bigwedge_{\substack{a < i < c\\ i \neq b}} \ov{\cvar_{i; a,b,c}}\Big) \rightarrow \hvar_{a, b, c} \quad \text{ for all } 2 \leq a < b < c \leq n\label{eq:holeDefClauses1}\\
%
\tikzmarkin{a}(12.0,-0.3)(-0.5,0.5)
  \orvar_{a, b, c} \land \orvar_{a, c, d} \rightarrow \orvar_{a, b, d} \quad \text{ for all } 2 \leq a < b < c < d \leq n\label{eq:signotopeClauses11}\\
  \tikzmarkend{a}\ov{\orvar_{a, b, c}} \land \ov{\orvar_{a, c, d}} \rightarrow \ov{\orvar_{a, b, d}} \quad \text{ for all } 2 \leq a < b < c < d \leq n \label{eq:signotopeClauses12}\\
%
\hfsetfillcolor{blue!10}
\hfsetbordercolor{blue!60!black}
\tikzmarkin{c}(12.0,-0.4)(-0.5,0.6)
  % \orvar_{1, b, c} \quad \text{ for all } 2 \leq b < c \leq n \label{eq:revLexClauses}\\
  \tikzmarkend{c}\left(\orvar_{\lceil \frac{n}{2} \rceil -1, \lceil \frac{n}{2} \rceil,\lceil \frac{n}{2} \rceil+1}, \ldots, \orvar_{2,3,4} \right) \succeq_{\text{lex}} \left(\orvar_{\lfloor \frac{n}{2}\rfloor +1,  \lfloor \frac{n}{2}\rfloor +2, \lfloor \frac{n}{2}\rfloor +3}, \ldots, \orvar_{n-2, n-1, n} \right)\label{eq:revLexClauses}\\
%
\hfsetfillcolor{orange!10}
\hfsetbordercolor{orange!60!black}
\tikzmarkin{d}(12.0,-0.3)(-0.5,0.5)
  \ov{\orvar_{a,b,c}} \land \ov{\orvar_{b,c,d}} \rightarrow \uvar_{a, c, d} \quad \text{ for all } 2 \leq a < b < c < d \leq n\label{eq:capDef}\\
  \orvar_{a, b, c} \land \orvar_{b, c, d} \rightarrow \vvar_{a, c, d} \quad \text{ for all } 2 \leq a < b < c < d \leq n \label{eq:cupDef}\\
  \uvar_{a,b,c} \land \ov{\orvar_{b,c,d}} \land \hvar_{a,b,d} \rightarrow \ufvar_{a, c, d} \quad \text{ for all } 2 \leq a < b < c < d \leq n,\; a+1<b\label{eq:capFDef}\\
  \uvar_{a, c, d} \rightarrow \ov{\orvar_{a,c,d}} \quad \text{ for all } 2 \leq a < c < d \leq n,\ a+1<c\label{eq:capDef2}\\
  \tikzmarkend{d}\vvar_{a, c, d} \rightarrow \orvar_{a,c,d} \quad \text{ for all } 2 \leq a < c < d \leq n,\; a+1<c\label{eq:cupDef2}\\
%
\hfsetfillcolor{red!10}
\hfsetbordercolor{red!60!black}
\tikzmarkin{e}(12.0,-0.5)(-0.5,0.5)
  \neg(\ufvar_{a,d,e} \land \orvar_{a, p, e}) \quad \text { for all } 2 \leq a < d < e \leq n, \; a < p < e, \; a+2 < d\label{eq:no6Hole1Below}\\
  \neg(\ufvar_{a,d,e} \land \ov{\orvar_{d, e, f}}) \quad \text { for all } 2 \leq a < d < e < f\leq n, \; a+2 < d\label{eq:no6Hole4Above}\\
  \neg(\uvar_{a,c,d} \land \vvar_{a, c', d} \land \hvar_{a,c,c'}) \quad \text{ for all } 2 \leq a < c < c' < d \leq n, \; a+1 < c\label{eq:no6Hole2Below1}\\
  \neg(\uvar_{a,c,d} \land \vvar_{a, c', d} \land \hvar_{a,c',c}) \quad \text{ for all } 2 \leq a < c' < c < d \leq n, \; a+1 < c'\label{eq:no6Hole2Below2}\\
  \tikzmarkend{e}\neg(\vvar_{a,c,d} \land \orvar_{c, d, e} \land \hvar_{a,c,e}) \quad \text{ for all } 2 \leq a < c < d < e \leq n, \; a+1 < c\label{eq:no6Hole3Below}
  \end{gather}
\end{spreadlines}
% \end{framed}
\caption{Encoding based on that of Heule and Scheucher for the Empty Hexagon Number~\cite{emptyHexagonNumber}. Each line determines a set of clauses. Unsatisfiability of the formula below for $n=30$ implies $h(6) \leq 30$, as detailed throughout the paper.}
\end{figure}


\subparagraph*{Variables.}
Let $S = (p_1, \ldots, p_n)$ be the list of points in canonical position.
We explain the variables of $\phi_n$
by specifying their values in the propositional assignment $\tau_S$
that is our intended model of $\phi_n$
corresponding to $S$. We then have:
\begin{itemize}
  \item
    For every $2 \leq a < b < c \leq n$, $\orvar_{a,b,c}$ is true
    iff $\sigma(p_a,p_b,p_c) = +1$.\footnote{
    Since the point set is in general position,
    we have $\neg \orvar_{a,b,c} \iff \sigma(p_a, p_b, p_c) = -1$.}

    The first optimization observes that orientations are antisymmetric:
    if $(p,q,r)$ is counterclockwise then $(q,p,r)$ is clockwise, etc.
    Thus one only needs $\orvar_{a,b,c}$ for ordered triples $(a,b,c)$,
    reducing the number of orientation variables by a factor of $3! = 6$
    relative to using all triples. The second optimization uses the \textbf{CCW-order} property of canonical positions:
    since all $\orvar_{1,a,b}$ are true, we may as well omit them from the encoding.

  \item
    Next, for every $a < b < c$ with $a < i < b$ or $b < i < c$,
    the variable $\cvar_{i;a,b,c}$ is true
    iff \lstinline|σPtInTriangle S[i] S[a] S[b] S[c]| holds.
    By \lstinline|σPtInTriangle_iff|, this is true exactly
    iff $p_i$ is inside the triangle $p_ap_bp_c$.

    The reason for assuming $(a,b,c)$ to be ordered is again symmetry:
    $p_ap_bp_c$ is the same triangle as $p_ap_cp_b$, etc.
    Furthermore thanks to the \textbf{$x$-order} property of canonical positions,
    if $p_i$ is in the triangle
    then $x(p_a) < x(p_i) < x(p_c)$.
    This implies that $a < i < c$,
    leaving one case distinction permuting $(i,b)$.

  \item
    For every $a < b < c$,
    $\hvar_{a,b,c}$ is true
    iff \lstinline|σIsEmptyTriangleFor S[a] S[b] S[c] S| holds.
    By a geometro-combinatorial connection analogous to ones above,
    this is true iff $p_ap_bp_c$ is a $3$-hole.

  \item
    Finally, one defines \emph{$4$-cap}, \emph{$5$-cap}, and \emph{$4$-cup} variables.
    For $a+1 < c < d$, $\uvar_{a,c,d}$ is true
    iff there is $b$ with $a < b < c$ with $\sigma(p_a,p_b,p_c) = \sigma(p_b,p_c,p_d) = -1$.
    $\vvar_{a,c,d}$ is analogous, except in that the two orientations are required to be counterclockwise.
    These are the $4$-caps and $4$-cups, respectively.
    The $5$-cap variables $\ufvar_{a,d,e}$
    are defined for $a+2 < d < e$.
    We set $\ufvar_{a,d,e}$ to true
    iff there exists $c$ with $a+1<c<d$
    such that $\uvar_{a,c,d}$, $\orvar_{c,d,e}$, and $\hvar_{a,c,e}$ are all true.

    Intuitively, $4$-caps and $4$-cups are clockwise and counterclockwise arcs of length $4$,
    respectively,
    whereas $5$-caps are clockwise arcs of length $5$ containing a $3$-hole.
    All three are depicted in [FIG].
\end{itemize}

\subparagraph*{Satisfaction.}
We now have to justify that the clauses of $\phi_n$
are satisfied by the intended interpretation $\tau_S$.
The variable-defining clauses~\labelcref{eq:insideClauses1,eq:insideClauses2,eq:holeDefClauses1,eq:capDef,eq:cupDef,eq:capFDef,eq:capDef2,eq:cupDef2}
follow essentially by definition combined with boolean reasoning.
The orientation properties~\labelcref{eq:signotopeClauses11,eq:signotopeClauses12}
have been established in the family of theorems \lstinline|theorem σ_propᵢ|.
The lexicographic ordering clauses~\labelcref{eq:revLexClauses}
follow from the \textbf{Lex order} property of canonical positions.
Thus we are left with clauses~\labelcref{eq:no6Hole1Below,eq:no6Hole4Above,eq:no6Hole2Below1,eq:no6Hole2Below2,eq:no6Hole3Below}.
These forbid the presence of certain $6$-holes.\footnote{
They are intended to forbid \emph{all} $6$-holes,
but we do not prove completeness.}

[WIP]
We illustrate the proof of clause~\labelcref{eq:no6Hole1Below}
This proof relies on two key lemmas:
first, that two arcs can be put together into \lstinline|σCCWPoints|
second, that any \lstinline|σCCWPoints| containing a $3$-hole
must form a $6$-hole.

\subparagraph*{QED.}
Having now shown that if $\phi_{30}$ were unsatisfiable,
our main result would follow,
we run a distributed computation to confirm the antecedent
and conclude the proof.

% file-local attic:

%This is captured by the following two fundamental asymmetries:
%\begin{lstlisting}
%  lemma σ_perm₁ (p q r : Point) : σ p q r = -σ q p r
%  lemma σ_perm₂ (p q r : Point) : σ p q r = -σ p r q
%\end{lstlisting}

% theorem σCCWPoints.emptyHexagon
%    (H : σCCWPoints [a, b, c, d, e, f]) (gp : Point.ListInGenPos S)
%    (hole : σIsEmptyTriangleFor a c e S.toFinset) (sp : [a, b, c, d, e, f] ⊆ S) :
%    σHasEmptyKGon 6 S.toFinset := by
