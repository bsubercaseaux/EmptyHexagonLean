Describe how we go from abstract formula over boolean variables
to a CNF using encoding programs.
CNF is usually large, and SAT solvers take the CNF in a format called DIMACS which is not human inspectable.
Thus usually generated by encoding program rather than constructed manually.
This DIMACS is fed to a SAT solver, which produces an UNSAT certificate
(usually in a format known as DRAT or LRAT).

Given this pipeline, we have two distinct concerns for correctness:
\begin{itemize}
    \item[(1)] \textbf{Is the CNF formula correct?}    
    We want to formally verify that our ``encoder'' program
    generates DIMACS output which
    correctly represent the abstract formula we wish to show UNSAT.
    \item[(2)] \textbf{Is the SAT solver output correct?}
    The UNSAT certificate must be checked for correctness
    against the DIMACS formula.
\end{itemize}

Prior work on (2) has produced highly trustworthy proof checkers
for SAT solver output (cake lpr citation, GRAT citation maybe?).
We rely on \texttt{cake\_lpr} for this step of the proof,
and trust that the DIMACS formula given to \texttt{cake\_lpr}
is unsatisfiable if this program terminates without error.

In contrast, we found very little prior work on (1).
Codel (cite Codel, any other papers Cayden found about this).

We built on (bryant citation, codel citation)
to develop tools for writing correct-by-construction encoding programs
against an abstract model of propositional formulas.
This infrastructure is a component of the fledgling \leansat{} library,
which is largely outside the scope of this paper.
Nonetheless this library enabled us to re-implement
Heule et al's encoding exactly as presented (TODO: is this true?).

Here we give a brief overview of how this section of the formalization proceeds.

\subsection{Variables}

The \leansat{} library allows formulas to be written over arbitrary variable sets.
As an example, for the Empty Triangle Theorem (see section 5) we use the following variables:
\begin{lstlisting}
inductive Var (n : Nat)
  | sigma  (a b c : Fin n)
  | inside (x a b c : Fin n)
  | hole   (a b c : Fin n)
\end{lstlisting}
TODO explain how the library handles converting these variables to \(\mathbb{N}\)
for the purposes of DIMACS.

\subsection{Abstract Propositional Formula}

Given that the encoding presented in Heule et al has many sets of clauses,
we divide the abstract formula into a few sub-formulas.
Here is the formula which defines the "is hole" variables:
\begin{lstlisting}
/--
  Triangle abc is a hole iff all x are not inside triangle abc.
-/
def holeDefs (n : Nat) : PropAssignment (Var n) → Prop := fun τ =>
  ∀ {a b c}, τ (Var.hole a b c) ↔
    (∀ x, a < x → x < c → x ≠ b →
      !τ (Var.inside x a b c))
\end{lstlisting}
The Lean statement here is clean and simple,
and nicely corresponds to the paper presentation.
We define the entire formula in around 100 lines of Lean.
TODO(JG): get exact numbers

\subsection{Encoding Program}

Now that we have defined an abstract formula over our variables,
we need to write a program which encodes this formula to DIMACS.
Again we divide the program into sub-programs,
corresponding exactly to how we divided up the formula.
Here is the signature of the encoder for the formula from the previous section:
\begin{lstlisting}
def holeDefClauses (n : Nat) : VEncCNF (Var n) Unit (holeDefs n) :=
  ...
\end{lstlisting}
The \texttt{VEncCNF} type is essentially a state monad,
where the state is the DIMACS output.
However, it is constrained to those programs which
verifiably encode the predicate \texttt{holeDefs n}.

In particular, if we run \texttt{holeDefClauses n},
we get DIMACS output which is satisfiable iff \texttt{holeDefs n} is satisfiable,
no matter what the particulars are in \texttt{holeDefClauses}.
This is expressed in the following theorem from \leansat{}:
\begin{lstlisting}
theorem toICnf_equisatisfiable [FinEnum ν] (v : VEncCNF L α P) :
  (∃ τ : PropAssignment IVar, τ |= v.val.toICnf.toPropFun) ↔
    (∃ τ : PropAssignment ν, P τ)
\end{lstlisting}
TODO(JG): cleanup in leansat

The encoding program, and its verification,
takes a few hundred lines of Lean.
TODO(JG) update when done rewriting

\subsection{Verifying UNSAT Proof}

We have generated a DIMACS-format formula which is satisfiable iff
our abstract encoding is satisfiable.
Ideally, at this point we would emit the formula,
run a SAT solver,
get a proof of unsatisfiability,
and then check this proof \textit{in the trusted Lean kernel}.
This approach would establish the formula unsatisfiable
with the same level of trust as all mathematics verified in Lean.

Unfortunately, such an approach is not feasible due to performance constraints.
The Lean kernel is simple and trustworthy,
but evaluates programs quite slowly,
orders of magnitude slower than executing compiled Lean programs.
The \(h(6) = 30\) result generated on the order of 100 TB of DRAT proof,
which is far, far beyond the range at which the kernel could feasibly check.

TODO(JG): explain our new trust story for this last leg of the journey.

\begin{lstlisting}
axiom cnfUnsat : ¬∃ τ, τ |= (theEncoding 30).toICNF
\end{lstlisting}
TODO(JG): make this statement reality
