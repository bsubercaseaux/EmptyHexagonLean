This formalization required approximately 300 hours of work over 3 months
by researchers with significant experience formalizing mathematics in Lean.
The final version of our proofs consists of approximately 4.7k lines of Lean code
(including comments/whitespace);
about $26\%$ consists of lemmas that should be moved to upstream libraries,
about $40\%$ develops the mathematical theory of orientations in plane geometry,
and the remaining $34\%$ (1550 LOC) validates the symmetry breaking and SAT encoding.
Many more lines of code were deleted or rewritten,
so these numbers should be taken with a grain of salt.

We substantially simplified the symmetry-breaking argument presented by Heule and Scheucher.
Beyond that, though, the correctness argument remains generally the same.
So what did formal verification contribute in this case?

We have substantially improved the level of trust in the new result by Heule and Scheucher.
Prior to formalization,
their result relied on the correctness of various components that are hard to validate manually,
while now we can assert with high confidence that their result is correct.
Beyond the main theorem presented here,
we demonstrated our approach for formalizing reductions to UNSAT,
and developed a broadly useful formal theory of orientations and plane geometry.

\subparagraph*{Future work.}
The geometric and combinatorial machinery developed for this formalization
will be useful in future formalizations of Erd\H{o}s-Szekeres style problems.
We believe automated reasoning will continue to be used in this area,
and hope future results can be verified with much less effort.
Beyond automated reasoning,
we hope to see formalizations of traditional pen and paper proofs in this area,
perhaps starting with a (formal) proof of the lower bound for the $h(6) = 30$ result.

We also believe work should be done towards improving the connection between verified SAT tools and ITPs.
Although we are confident that our results are correct,
the trust story at the connection point clearly has room for improvement,
starting with some simple automation for running verified SAT toolchains from Lean.
