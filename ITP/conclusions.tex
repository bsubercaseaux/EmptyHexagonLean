We have proved the correctness of the main result of Heule and Scheucher~\cite{emptyHexagonNumber},
implying $h(6) \leq 30$.
Given that the lower bound $h(6) > 29$ can be checked directly (see \cite{emptyHexagonNumber}),
we conclude the result $h(6) = 30$ is indeed correct.
We believe this work puts a \emph{happy ending} to
one line of research started by Klein, Erd\H{o}s and Szekeres in the 1930s.
Prior to formalization, the result of Heule and Scheucher
relied on the correctness of various components of a highly sophisticated encoding
that are hard to validate manually.
We developed a significant theory of combinatorial geometry
that was not present in~\textsf{mathlib}.
Beyond the main theorem presented here,
we showed how our framework can be used for other related theorems
such as $g(6) = 17$,
and we hope it can be used for proving many further results in the area.

Our formalization required approximately 300 hours of work over 3 months
by researchers with significant experience formalizing mathematics in Lean.
The final version of our proofs consists of approximately 4.7k lines of Lean code;
% (including comments/whitespace);
about $26\%$ are lemmas that should be moved to upstream libraries,
about $40\%$ develops the theory of orientations in plane geometry,
and the remaining $34\%$ (1550 LOC) validates the symmetry breaking and SAT encoding.
% Many more lines of code were deleted or rewritten,
% so these numbers should be taken with a grain of salt.
%  (BS: commented to save space)

We substantially simplified the symmetry-breaking argument presented by Heule and Scheucher,
and derived in turn from Scheucher~\cite{scheucherTwoDisjoint5holes2020}.
Moreover, we found a small error in their proof,
as their transformation uses the mapping $(x, y) \mapsto (x/y, -1/y)$,
and incorrectly assumes that $x/y$ is increasing for points in CCW-order,
whereas only the slopes $y/x$ are increasing.
Similarly, we found a typo in the statement of the \textsf{Lex order} condition
that did not match the (correct) code of Heule and Scheucher.
Our formalization corrects this.

\subsection*{Future Work}
We hope to formally verify the result $h(7) = \infty$ due to Horton~\cite{hortonSetsNoEmpty1983},
and other results in Erd\H{o}s-Szekeres style problems.

We also want to improve the trust story of importing ``cube and conquer''-style results into an ITP.
Importing these kinds of proofs is a significant engineering task
when the proof certificate is hundreds of terabytes in size,
as it was for this result (see \Cref{sec:running_cnf}).
Although we are confident that our results are correct,
more work needs to be done to strengthen the trust in this connection point.
