We have proved the correctness of the main result of Heule and Scheucher~\cite{emptyHexagonNumber},
implying $h(6) \leq 30$.
Given the lower bound $h(6) > 29$ can be checked directly (see \cite{emptyHexagonNumber}),
we conclude the result $h(6) = 30$ is indeed correct.
We believe this work puts a \emph{happy ending} to
one line of research started by Klein, Erd\H{o}s and Szekeres in the 1930s.
Prior to formalization, the result of Heule and Scheucher
relied on the correctness of various components of a highly sophisticated encoding
that are hard to validate manually.
We developed a significant theory of combinatorial geometry
that was not present in~\textsf{mathlib}.
Beyond the main theorem presented here,
we showed how our framework can be used for other related theorems
such as $g(6) = 17$,
and we hope it can be used for proving many further results in the area.

Our formalization required approximately 300 hours of work over 3 months
by researchers with significant experience formalizing mathematics in Lean.
The final version of our proofs consists of approximately 4.7k lines of Lean code;
% (including comments/whitespace);
about $26\%$ consists of lemmas that should be moved to upstream libraries,
about $40\%$ develops the mathematical theory of orientations in plane geometry,
and the remaining $34\%$ (1550 LOC) validates the symmetry breaking and SAT encoding.
% Many more lines of code were deleted or rewritten,
% so these numbers should be taken with a grain of salt.
%  (BS: commented to save space)

We substantially simplified the symmetry-breaking argument presented by Heule and Scheucher,
and derived in turn from Scheucher~\cite{scheucherTwoDisjoint5holes2020}.
Moreover, we found a small error in their proof,
as their transformation uses the mapping $(x, y) \mapsto (x/y, -1/y)$,
and incorrectly assumes that $x/y$ is increasing for points in CCW-order,
whereas only the slopes, $y/x$, is increasing.
Similarly, we found a typo in the statement of the \textsf{Lex order} condition
that did not match the (correct) code of Heule and Scheucher.
This has been corrected in our formalization.

In terms of future work,
we hope to formally verify the result $h(7) = \infty$ due to Horton~\cite{hortonSetsNoEmpty1983},
and other results in Erd\H{o}s-Szekeres style problems.
We also believe a key challenge for the community
is to improve the connection between verified SAT tools and ITPs,
which presents a significant engineering challenge
for proofs which are hundreds of terabytes long (as in this result).
Although we are confident that our results are correct,
the trust story at this connection point has room for improvement.
